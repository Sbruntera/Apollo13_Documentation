\section{Beschreibung der Bodenstation}
\subsection{Desktop Applikation}
\subsection{Einleitung}
In diesem Teil der Dokumentation werden wir die Bodenstation vorstellen, welche als Datenempfänger und als Datenverarbeitungsplattform fungiert. \\
Die Bodenstation wurde von Robin Bley, Marc Huisinga und Kevin Neumeyer entwickelt.

Die zentrale Aufgabe der Bodenstation ist es, die Daten, welche vom Satelliten gesammelt werden, zusätzlich sicher am Boden zu speichern, sollte der Satellit und damit auch die lokal gespeicherten Daten verloren gehen. \\
Zusätzlich zur Datensicherung erfüllt die Bodenstation die Aufgabe, die empfangenen Daten auf verschiedene Arten zu visualisieren und somit dem Nutzer direkt während der Datenübertragung die Möglichkeit zu verschaffen, die Daten zu beobachten und diese zu analysieren. \\
Die Bodenstation ermöglicht es außerdem, dass gesicherte Daten auch nach der Datenübertragung noch betrachtet und analysiert werden können. \\
Unser Ziel bei der Entwicklung der Bodenstation war es, eine modulare und anpassbare Plattform zu entwickeln, welche nicht nur mit unserem Satelliten, sondern mit vielen verschiedenen Satelliten genutzt werden kann, ohne dass ein großer Konfigurationsaufwand besteht. \\
Um dies zu ermöglichen, haben wir die Bodenstation in mehrere Dimensionen skalierbar entwickelt, was es im Endeffekt sehr einfach macht, neue Satelliten und verschiedene Übertragungsprotokolle zur Bodenstation hinzuzufügen.

\subsubsection{Verwendete Komponenten}
Zum Erreichen unserer Ziele haben wir verschiedene Komponenten verwendet, welche einerseits der Datenvisualisierung und -analyse dienen, andererseits aber auch der Entkopplung und Skalierbarkeit für die Entwicklung.

Für die Bodenstation haben wir folgende Komponenten verwendet:
\begin{description}
	\item[\href{http://www.oracle.com/technetwork/java/javase/downloads/jdk8-downloads-2133151.html}{Java}] ist eine objektorientierte Programmiersprache. Diese wurde verwendet, da jedes unserer Gruppenmitglieder damit vertraut ist. Die Version Java 8 wurde verwendet um mächtige funktionale Features zu nutzen.
	\item[\href{https://netbeans.org/features/platform/}{Netbeans Platform}] das die Möglichkeit bietet, einfach eine integrierte, modulare und entkoppelte GUI-Applikation auf Basis von Java Swing zu entwickeln.
	\item[\href{http://junit.org/}{JUnit}] ist ein Framework, welches zum Erstellen von automatisierten Softwaretests dient.
	\item[\href{http://fazecast.github.io/jSerialComm/}{JSerialComm (zum Start des Projektes noch serial-comm)}] ist eine Bibliothek, welche das Auslesen serieller Schnittstellen ermöglicht 
	\item[\href{http://worldwind.arc.nasa.gov/java/}{NASA World Wind}] ist eine Software, welche Satelliten- und Luftbilder auf einem virtuellen Erdball darstellt. Daten der Bodenstation werden mittels dieser Software in Relation zur Höhe in Echtzeit visualisiert.
	\item[\href{http://jchart2d.sourceforge.net/}{JChart2D}] ist eine Grafik-Bibliothek, welche zur grafischen Visualisierung von Daten dient. Mithilfe dieser Bibliothek werden zweidimensionale Graphen erzeugt, welche empfangene Daten des Satelliten, in Relation zur Zeit oder anderen Daten, in einem Graphen darstellt
	\item[\href{http://www.json.org/}{JSON (JavaScript Object Notation)}] ist ein Datenformat, welches zum Austausch von Daten zwischen Anwendungen angewand wird. JSON ermöglicht es Daten in Textform zu speichern und sie wieder zurück in ihre uhrsprüngliche Form zu interpretieren. Dieses Datenformat wird in der Bodenstationssoftware genutzt um Daten mit dem Sateliten auszutauschen, zu loggen, zu exportieren und zu importieren.
	\item[\href{http://git-scm.com/}{Git}] ist eine freie Versionsverwaltungssoftware. Diese Software verwendeten wir um die Versionen unserer Software auszutauschen und zu verwalten.
\end{description}

\subsection{Funktionen}
\subsubsection{Nutzerfreundlichkeit}
Die Bodenstation wurde so entwickelt, dass der Nutzer der Bodenstation sich nicht um Implementationsdetails scheren muss und die Bodenstation als zentralen Empfänger für Daten von seinem Satelliten nutzen kann, ohne dabei etwas anderes als die grafische Benutzeroberfläche zu verwenden. 

Ein wichtiger Faktor im Bereich der Nutzerfreundlichkeit ist die dynamische Benutzeroberfläche, welche es dem Nutzer erlaubt, verschiedene GUI-Komponenten in Teilpanels innerhalb der Applikation anzuzeigen und umzustellen. Diese Dynamik ermöglicht es dem Nutzer, die Benutzeroberfläche, welche für die Analyse seiner Daten am Besten ist, mithilfe der verschiedenen Panels einzustellen.

Ein weiterer, wichtiger Faktor ist, dass der Nutzer die Bodenstation so anpassen kann, wie es für die Datenübertragung seines Satelliten am Besten ist. Alle Teilkomponenten der Datenübertragung sind über die grafische Benutzeroberfläche austauschbar, was es sehr einfach macht, die Bodenstation auf die Datenübertragung des jeweiligen Satelliten anzupassen.

Alle Visualisierungskomponenten sind zudem intuitiv aufgebaut, sodass es nicht kompliziert ist, sich seine Daten mithilfe der Visualisierungskomponenten anzusehen. Die Anzeige über den 3D-Globus erlaubt es beispielsweise, den Globus beliebig zu bewegen und die Position auf dem Globus zu verändern, während der Graph es erlaubt, dass die Axen des Graphen beliebig ausgetauscht werden können.

\subsubsection{Erweiterbarkeit}
Bei der Entwicklung der Bodenstation haben wir darauf geachtet, dass die Bodenstation auf einer skalierbaren Architektur aufgebaut ist. Dies ermöglicht es, leicht neue Module und Funktionalitäten zur Bodenstation hinzuzufügen, ohne dabei besonders viel Code abzuändern. Auf die folgenden Weisen ist die Bodenstation skalierbar:
\begin{itemize}
	\item Neue GUI-Komponenten können sehr leicht hinzugefügt werden. Das Erstellen und Einbinden eines GUI-Komponenten umfasst lediglich die Erstellung eines neuen Netbeans-TopComponents.
	\item Unterschiedliche Satelliten können ohne eine erneute Kompilierung hinzugefügt und verändert werden, indem man die Konfigurationsdateien der Bodenstation anpasst, welche zur Laufzeit von der Bodenstation geladen werden.
	\item Neue Konfigurationsformate können leicht hinzugefügt werden, indem man die neue Konfigurationsimplementation unter einem Interface in der Applikation hinzufügt.
	\item Es ist ohne viel Aufwand möglich, die verschiedenen Datenquellen, aus denen Daten bezogen werden, auszutauschen. Möchte man also die Daten von einer anderen Quelle als einem USB-Port beziehen, so ist dies leicht zu implementieren, indem man ledigliche eine neue DataSource implementiert und zur Applikation hinzufügt.
	\item Verschiedene Datenübertragungsformate können ebenfalls ohne viel Aufwand hinzugefügt werden, indem neue Formate unter dem DataFormat-Interface implementiert werden. Die Applikation ist also nicht auf JSON als Übertragungsformat limitiert.
	\item Die verschiedenen Datenempfänger können auch leicht angepasst werden, indem man neue Datenempfänger unter dem Receiver-Interface implementiert, wodurch die verschiedenen Logging-Formate erweitert werden können.
	\item Daten können aus beliebigen Dateiformaten importiert werden, was ebenfalls leicht erweiterbar ist, indem man neue Import-Dateiformate über das Importer-Interface implementiert.
	\item Export-Formate können leicht erweitert werden, indem man neue Export-Formate unter dem Exporter-Interface implementiert.
\end{itemize}

\subsubsection{Features}
Da die Software der Bodenstation auf dem Framework Netbeans Plantform basiert, lassen sich einzelne graphische Module kombinieren, welche sich per drag and drop verschieben lassen. Die Größe und Position dieser Module und des gesamten Frames lassen sich beliebig verändern. Des weiteren bietet die Software die Möglichkeit, Daten von verschiedenen Satelliten zu empfangen. Empfangene Daten lassen sich mittels der graphischen Oberfläche in Graphen anzeigen, welche sich verschieden kombinieren lassen. Außerdem lassen sich die empfangenen Daten zwischenspeichern und anschließend in verschiedene Dateiformate exportieren oder live in einer Datei loggen. Diese exportierten oder geloggten Daten lassen sich anschließend wieder einlesen und anzeigen. CSV, TXT, JSON, KML und PNG sind Dateiformate, welche exportierbar sind. Davon lassen sich exportierte CSV- und JSON Dateien wieder einlesen und visualisieren. TXT-Dateien werden formatiert und somit gut leserlich für den Nutzer exportiert, während exportierte KML-Dateien per Google Earth geöffnet und graphisch visualisiert werden können. Ein weiteres Feature der Bodenstationsoftware ist die Datenvisualisierung per Nasa World Wind als Modul in der graphischen Oberfläche. Diese Visualisierung zeigt den Flug des Satelliten auf einem virtuellen Globus, mittels Satelliten- und Luftbilder, und zeigt auf jeder gemessenen GPS-Koordinate die gemessenen Werte der Sensoren des Satelliten. Unter anderem bietet dieses Modul der Software die Möglichkeit, an die virtuelle Erdkugel heranzuzoomen und einzelne Elemente dreidimensional darzustellen. Darstellungen mittels dieses virtuellen Erdballs sind sowohl in Echtzeit mittels eines Streams vom Satelliten als auch als Import aus einer Datei möglich.


\subsubsection{Architektur}
Insgesamt ist die Architektur der Applikation um das GUI-Framework Netbeans Platform aufgebaut, da es die Nutzung von bestimmten Architekturen einfacher macht, mit Netbeans Platform zu arbeiten und alle Features von Netbeans Platform zu nutzen.

Insgesamt ist die Applikation in Module und Pakete aufgeteilt. Während normale Java-Projekte normalerweise lediglich in Pakete aufgeteilt sind, werden in Netbeans Platform die einzelnen Komponenten in Module aufgeteilt. Jedes Modul verhält sich hierbei wie ein einzelnes Projekt, welches dann vom Hauptprojekt eingebunden wird. Dies fördert generell die Wiederverwendung der einzelnen Module, da sich Entwickler darüber Gedanken machen müssen, wie die einzelnen Module in verschiedenen Umgebungen verwendet werden können. \\
Den Aufbau der Netbeans Platform Modularchitektur ist auch im Anhang unter Abbildung \ref{nbp_modularchitektur} zu finden.

Anfänglich haben wir jeden einzelnen Teilkomponenten in ein Modul ausgelagert, was jedoch zu einer Menge Merge-Konflikten geführt hat, da Netbeans Platform für jedes einzelne Modul eigene Konfigurationsdateien generiert, über welche man nur schwer den Überblick behalten kann. Diese anfängliche Architektur wurde schlussendlich in eine Architektur mit nur drei Modulen umgewandelt: API, Core und GUI. \\
Das Core-Modul enthält die Programmlogik, welche sich hauptsächlich mit der Verarbeitung der Daten innerhalb der Input-Pipeline beschäftigt. \\
Das GUI-Modul enthält die verschiedenen GUI-Komponenten, welche sowohl Teil der allgemeinen Benutzeroberfläche sind, als auch Visualisierungskomponenten darstellen. \\
Innerhalb des API-Moduls befinden sich Interfaces und Utility-Klassen, welche sowohl vom Core-Modul als auch vom GUI-Modul genutzt werden. Das Core-Modul implementiert hierbei die Interfaces aus dem API-Modul, während das GUI-Modul die Programmlogik im Core-Modul lediglich entkoppelt über die Interfaces des API-Moduls anspricht. \\
Diese Architektur zeigt bereits, dass das GUI-Modul vom Core-Modul entkoppelt ist. Diese Entkopplung trägt stark zu der Erweiterbarkeit der Applikation bei. Die Architektur ähnelt hierbei der standardmäßigen ``Model, View, Controller''-Architektur, jedoch scheint innerhalb der Architektur kein Controller vorhanden zu sein. Auf den ersten Blick gesehen scheint es so, als spreche das GUI-Modul das Core-Modul direkt über das API-Modul an, jedoch sorgt Netbeans-Platform dafür, dass dem nicht so ist. Die von Netbeans Platform bereitgestellten Lookups, welche eine weitere Ebene der Entkoppelung darstellen, erfüllen in der Applikation die Aufgabe des Controllers. Durch die Lookups wird innerhalb des GUI-Moduls eine passende Klasse innerhalb des Core-Moduls über das Interface der Klasse im API-Modul geladen. Dank den Lookups und den Interfaces im API-Modul besteht absolut keinerlei Kopplung zwischen den einzelnen Klassen und Modulen: Die GUI kennt das Model nicht und das Model kennt die GUI nicht. \\
Die Modularchitektur der Bodenstation ist ebenfalls im Anhang unter Abbildung \ref{station_modularchitektur} zu finden.

Weitergehend relevant ist die Architektur der Input-Pipeline, welche in der Bodenstation dafür zuständig ist, die empfangenen Daten zu verarbeiten, bevor sie an die jeweiligen Komponenten zur Endbehandlung der Daten weitergeleitet werden. Insgesamt wandelt die Input-Pipeline die aus einem Datenstream empfangenen Daten in einen Map-Datentypen um, welcher genau einem Datensatz entspricht, und leitet die Daten an alle registrierten Empfänger weiter. Die Schlüssel der Map entsprechen den Schlüsseln der Datenübertragung und die Werte der Map entsprechen den Daten, welche zu dem jeweiligen Schlüssel gehören. Zusätzlich zu dieser Datentransformation sorgt die Input-Pipeline ebenfalls dafür, dass fehlerhafte und fehlende Daten herausgefiltert und über die zuletzt erhaltenen Daten abgeflacht werden. \\
Die Input-Pipeline ist Push-basiert, was bedeutet, dass jeder Komponent immer die verarbeiteten Daten an den nächsten Komponenten weitergibt, damit zwischen den einzelnen Komponenten nicht gepollt werden muss. Die Input-Pipeline besteht insgesamt aus vier Komponenten. Am Anfang der Input-Pipeline steht eine DataSource, welche die Daten von einer beliebigen Datenquelle via Polling bezieht und die ausgelesenen Daten an ein DataFormat weitergibt. DataFormat sorgt dafür, dass der von einer DataSource empfangene String mit dem jeweiligen Übertragungsformat, zum Beispiel JSON, geparst wird, um so einen Datensatz als Map zu erzeugen. DataFormat leitet die geparsten Daten dann an einen DataProvider weiter, welcher die fehlenden Daten herausfiltert, abflacht, und an alle bei ihm registrierten Empfänger weiterleitet. Bei der Abflachung der Daten merkt sich der DataProvider immer die zuletzt empfangenen Daten und ergänzt fehlende Daten mit den zuletzt empfangenen Daten. Sind noch keine Daten empfangen worden, so werden die Daten mit Default-Werten ergänzt. \\
Insgesamt wird die Input-Pipeline von einer DataPipeline umschlossen, welche die Input-Pipeline zusammenbaut, die verschiedenen Komponenten austauscht, die verschiedenen Empfänger beim DataProvider registriert und allgemein als Schnittstelle zu den verschiedenen Empfängern und dem GUI-Modul fungiert. \\
Der Aufbau der Input-Pipeline ist ebenfalls im Anhang unter Abbildung \ref{inputpipeline} zu finden.

\subsubsection{Tests}
\paragraph{Automatisierte Tests}
Einzelne Komponenten der Software werden mittels automatisierten Tests per JUnit getestet. Dabei werden bei den jeweiligen Export-Komponenten jeweils Testdaten in das jeweilige Format exportiert. während dessen kann überprüft werden ob sich die Komponente, bei der Übergabe verschiedener Parameter, wie geplant verhält. Außerdem kann geprüft werden, ob die erzeugten oder veränderten Dateien wie geplant aussehen. Darüber hinaus wurde für jeden Import-Komponenten ein automatisierten Test geschrieben, welche den Komponenten auf das verhalten bei verschiedenen Parametern überprüft. Des weiteren werden Daten erzeugt, welche zunächst mit dem jeweiligen Komponenten exportiert werden und anschließend mit dem passenden Komponenten importiert werden. Dabei wird geprüft, ob sich die importierten Daten von den ursprünglichen Daten unterscheiden.

\paragraph{Manuelle Tests}

\subsubsection{Nutzeranleitung}
Die hier vorgestellte Nutzeranleitung repräsentiert die Nutzeranleitung des Produktes, wenn es komplett fertiggestellt ist. \\
Da dies noch nicht komplett der Fall ist, kann es Abweichungen zwischen der Nutzeranleitung und dem für das P5 abgegebene Produkt geben.

\paragraph{Datenempfang}
Um Daten in Echtzeit zu empfangen, muss eine Verbindung zu einem Satelliten aufgebaut werden. Hierzu muss generell zu allererst ein Satellit erstellt werden. Hierfür wählt man unter dem Menüpunkt ``File'' den Unterpunkt ``Satellites'' aus. Dort ist es möglich, unter ``Add'', einen Satelliten hinzuzufügen. Um eine Satelliten zu laden, wählt man im selben Unterpunkt ``Manage'' aus. Dort wählt man den Satelliten aus, von welchem man Daten empfangen will. Möchte man den Datenempfangen beginnen, dann klickt man auf das ``Start''-Icon in der Toolbar, was die Datenübertragung startet. Ab hier können verschiedene Visualisierungen geöffnet werden, um die empfangenen Daten darzustellen.
\paragraph{Datenimport}
Um Daten aus einer Datei zu importieren, wird im Menüpunkt ``File'' der Untermenüpunkt ``Import'' ausgewählt. Anschließend wird eine Datei ausgewählt, welche importiert werden soll. Mit der Bestätigung werden die Daten dieser Datei eingelesen.
\paragraph{Datenexport}
Für das Exportieren der Daten gilt, dass alle aktuell geladenen Daten exportiert werden. Darunter fallen entweder zwischengespeicherte Daten einer Liveübertragung, oder Daten, welche aus einer Datei importiert werden.
Zum Exportieren der Daten wird im Menüpunkt ``File'' der Untermenüpunkt ``Export'' ausgewählt. Unter diesem Menüpunkt ist das Datenformat wählbar, in welches die gesammelten Daten gespeichert werden. Anschließend ist ein Pfad und ein Name wählbar, unter dem die Datei gespeichert wird. Mit der Bestätigung werden die Daten exportiert.
\paragraph{Datenweiterleitung}
Per Klick auf das Icon des Servers in der Toolbar wird ein Hotspot-Server gestartet. Dieser sendet empfangene Daten in Echtzeit an alle Clients weiter.
\paragraph{Oberflächenpersonalisierung}
Der Oberfläche können einzelne Komponenten hinzugefügt und entfernt werden. Diese Komponenten können unterschiedlich angeordnet werden. Um Komponenten, wie z.B einen Graphen, hinzuzufügen, wird entweder eine Graphenvisualisierung oder eine Kartenvisualisierung hinzugefügt. Um einen der bestehenden Komponenten zu entfernen wird das Kreuz angeklickt, welches sich am Tab des Komponenten befindet. Per ``drag and drop''  können diese Komponenten neu angeordnet werden, dazu muss der Tab des Komponenten ausgewählt werden. In verschiedenen Bereichen können Komponenten angeheftet oder verschoben werden. Außerdem können diese übereinander verlagert werden um sie in verschiedenen Tabs und in dem selben Bereich zu verwalten.
\paragraph{Kartenvisualisierung}
Die Kartenvisualisierung startet über den Untermenüpunkt ``Map Visualization'' im Menüpunkt ``Window''. Geladene Werte werden dort angezeigt. Einzelne Messpunkte sind mit einem Punkt gekennzeichnet und mit einer Linie verbunden, was den Flugweg des Satelliten anzeigt.
\paragraph{Graphenvisualisierung}
Um einen Graphen zu erzeugen, wählt man unter dem Menüpunkt ``Window'' - ``Graph Visualization'' aus. Anschließend wird in der Oberfläche ein Graph erzeugt. Die Achsen des Graphs sind mit Sensorwerten belegbar. Um Belegung der Achsen zu verändern, wählt man an den Achsen den jeweiligen Sensor aus und drückt den Button ``Refresh axes''.

\paragraph{Textdarstellung}
Zusätzlich können die empfangenen Daten auch direkt dargestellt werden, indem man den Menüpunkt ``Window'' - ``Text Stream'' auswählt.

\paragraph {Tabellendarstellung}
Die empfangenen Daten können ebenfalls in einer Tabelle dargestellt werden, welche über den Menüpunkt ``Table Visualization'' geöffnet werden kann.

\paragraph{Fenster zurücksetzen}
Die Anordnung der Komponenten der Oberfläche kann im Menüpunkt ``Window'' unter ``Reset Windows'' zurückgesetzt werden.

\paragraph{Beenden des Programms}
Um das Programm zu beenden, gibt es zwei Möglichkeiten. Zum einen wird das Programm beendet, wenn das Kreuz am oberen rechten Rand der Oberfläche angeklickt wird. Zum anderen kann das Programm über den Menüpunkt ``File'' geschlossen werden, in dem man dort ``Exit'' auswählt.

\subsubsection{Kosten-/Nutzenanalyse}


\subsection{Spezifische technische Beschreibungen}
Datenausglättung
...
...

\subsection{Realisierung}
Während der Realisierung der Bodenstation kam es zu einigen Komplikationen. Zum einen wurde die Softwarearchitektur während der Implementationsphase geändert, so dass verschiedene Komponenten wie zum Beispiel Export- und Importkomponenten mehrfach realisiert wurden. Diese Architekturveränderung wurde vorgenommen und gewisse Komplikationen zu beseitigen, welche die Modulare Strukturierung von Netbeans Plattform mit sich bringt. Wir starteten mit einer Architektur, welche jede wichtige Komponente als Modul benennt. Diese Architektur brachte zum einen das Problem, dass Abhängigkeiten zwischen Modulen nur in eine Richtung stattfinden kann. Die neue Architektur unterscheidet lediglich zwischen den Modulen API, GUI und Core. Des weiteren wurden verwendete Datentypen innerhalb der Software während der Implementationsphase geändert, so dass zusätzlich alle Komponenten, welche mit der Datenverarbeitung Zutun haben geändert werden mussten. Darunter vielen die gesamten Komponenten des Imports, Exports und der Live-Datenverarbeitung. Die Veränderung der benutzten Datentypen wurde von Long, String und Double zu einzig Double geändert, da alle Daten, welche von kompatiblem Satelliten in Double dargestellt werden können. Diese Änderung im Programmcode hat zwar einiges an Arbeit gekostet, doch der Umfang des Programmcodes wurde deutlich verringert und die Performance gesteigert.


\subsection{Android Applikation}
\subsubsection{App Übersicht}
Auch dieses Jahr haben wir uns für die Erstellung einer App entschieden. Die Hauptaufgaben der App soll sein, die Datenpakete, die von der Bodenstation über einen Hotspot gesendete werden, zu empfangen und Live, in einem passenden Graphen anzeigen zu können. Dabei legen wird Wert drauf, das es möglich ist, alle Werte die gesendet werden einzeln oder in Gruppen darzustellen sind, um sie anhand der Daten vergleichen zu können. Zusätzlich haben wir uns überlegt, falls die Zeit reicht, auch die Werte in einem Balkendiagramm anzeigen zulassen. Dabei sollen die Werte nicht einfach, wie im Livegraphen live angezeigt werden, sondern es soll die Differenz ausgerechnet werden, von dem höchsten und niedrigsten Punkt im Flug. Diese Differenzen sollen dann Grün für positive Veränderung und Rot für negative Veränderung dargestellt werden. Neben her sollen ebenfalls einige Optionsmöglichkeiten vorhanden sein, um die Graphen nach den Individuellen wünschen des Nutzern zugestalten.

\subsubsection{Plattform und Komponenten}
Für die Entwicklung der App haben wir uns der Plattform von Androidstudio bedient. Durch eine übersichtlich Anordnung von Fenstern und Struktur ist dies für uns ein ideales Programmierumgebung. Androisstudio ist auf Java ausgelegt, das wiederum eine groß Zahl von Features mit sich bringt. Aber allein auf dieses Programm konnten wir uns nicht verlassen, es brauchte noch zwei weiter Komponenten, um den Graphen anzeigen zu lassen und das JSON zu entpacken. Dazu haben wir uns der AndoridPlot-Libary und einer der Liberies über JSON zur Hilfe genommen.

\subsubsection{Funktionen}
Für die App haben wir uns für folgende Funktionen vorgenommen:
\begin{itemize}
	\item Anzeigen der Werte in einem Livegraphen
	\item Verwaltung der angezeigten Werte im Graphen während der Laufzeit
	\item Anzeigen von Differenzen von Werten im Balkendiagramm in Rot und Grün
	\item Manuelle Start/Stop Funktion für den Balkendiagramm
	\item Einstellung zur Geschwindigkeit des Graphen
	\item Einstellung wie viele Werte gleichzeitig angezeigt werden sollen
	\item Möglichkeit alle Funktionen mit einem Debugger zu testen
\end{itemize}

\subsubsection{Nutzeranleitung}
Um die Team-Gamma App nutzen zu können, brauche der Benutzer unser apk.-Datei. Diese werden wir unter den gegebenen unterständen, vor dem Raketen Start, Vorort verteilen. Um diese dann zu installieren zu können, braucht es ein geeigneten Dowloadmanager der es erlaubt die apk von dort aus zu installieren. Nach der erfolgreichen Installation der App kann man nun durch betätigen des Icons der App, diese dann, ganz wie gewohnt, starten. Davor sollte drauf geachtet werden, das um die Daten von der Bodenstation empfangen zu können, muss sich der Nutzer sich vorher in den zur Verfügung gestellten Hotspot einloggen. Dieser wird ohne Password in der nähe unserer Bodenstation erreichbar. Wir wollen aber noch mal ausdrücklich drauf Hinweisen, das dieser Hotspot keine Internet Verbindung bietet, sondern nur auf lokaler Basis arbeitet.
Nach öffnen der App sollte nach kurzer Zeit das Menü erscheinen, das die App in 3 Sektionen unterteilt: 

	\paragraph{Livegraph}
	Beim an tippen des Graphen-Icons erscheint ein graues Rechteck mit einigen Linien. Bei standardmäßigen Einstellungen, sollte fürs erste keine Werte angezeigten werden. Nur falls sich das Gerät in der nähe unseres zur Verfügung gestellten Hotspot befindet und genug Empfang hat werden nach einigen Sekunden automatisch Werte von der Bodenstation live angezeigt. Diese sollten aber während der Vorbereitungsphase nur gerade Linien darstellen. Nun kann, durchs  drücken der Menütaste des Smartphones, eine Fenster erscheinen lassen, das dazu genutzt werden kann, explizite Werte anzeigen zu lassen. Dabei ist zu beachten, das auch mehrere Werte gleichzeitig im Graph darstellbar sind.
	\paragraph{Balkengraph}
	Im Balkengraph angekommen sollte dort ebenfalls ohne den Debugger nichts zu sehen sein. Im Normalfall sollte, wenn die Verbindung mit der Bodenstation steht, der Graph starten, wenn der CanSat über die Marke von 1000 Meter ist. Falls durch irgendwelche Defekte der Racket diese Marke nicht zu erreichen sei, kann man Manuelle durch betätigen des Menüknopf des Smartphones ein Fenster erscheinen lassen, wo man diesen manuell starten und stoppen kann. Der Graph wird dann den ersten Wert als ersten Punkt nehmen und den letzte gesendet Punkt als zweiten. Die Differenz wird dann positiv in Grün oder negativ in Rot dargestellt. Beim verlassen des Balkendiagramms wird dieser im Hintergrund weiter ausgeführt. Bei einer Marke von ungefähr 0 Metern, sollte der Graph sich selber beenden und die Werte solange anzeigen, bis sie nicht mehr benötigt werden. Beim Beenden des Programm sind diese aber unwiderruflich verloren!
	\paragraph{Optionen}
	Die Option in der App geben die Möglichkeit den Livegrafen schneller oder langsamer zu machen. Außerdem wird dort auch ermöglicht, wie viele Werte gleichzeitig angezeigt werden sollen. Für das Testen und Anzeigen von Werten gibt es dort zusätzlich auch einen Knopf um den Debugger einzuschalten.
	
