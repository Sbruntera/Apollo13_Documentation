\subsection{Bergungssystem}
Für unser Landesystem haben wir uns entschieden, unseren eigenen Fallschirm zu bauen. Die Hauptaufgabe ist es, eine weiche Landung auf dem Boden zu garantieren. Unsere Vorgabe war, dass der Fallschirm und die Dose eine Fallgeschwindigkeit von 15 Meter/Sekunde haben soll. Leider sind unsere Testfallschirme, die wir auch schon im Vorjahr genutzt haben, für eine deutlich geringer Fallgeschwindigkeit ausgelegt. Dies haben wir uns zum Anlass genommen eine neue Fallschirmart zu testen. Die Idee dabei ist, dass die acht Schnüre, welche den Fallschirm an der Dose befestigen, mit sehr luftdurchlässigem Stoff, sogenannter Gaze, ersetzt werden. Wir erhoffen uns dadurch eine stabilere Lage in der Luft und ein fortschrittlicheres Design.

\subsubsection{Berechnungen}
Für den Bau des Fallschirm wissen wir bereits, dass dieser mit v = 15m/s fallen soll. Außerdem haben wir einen bereits berechneten Strömungswiderstandskoeffizient (Cw) von 1,33. Für die Berechnung des Fallschirms wurde folgende Formel verwendet:
\[
Fw = Cw*\frac{1}{2}*roh*v^2*A
\]
Fw ist die Strömungswiderstandskraft. Diese kann ermittelt werden, indem man die Fallwiederstandskarft einsetzt.
\[
Fw = m * g = 350g * 9,81\frac{m}{s^2} = 3,3433\frac{m}{s^2}*Kg
\]
Um die Größe des Fallschirms zu berechnen, kann man nun durch Einsetzen in die erste Formel diese nach A umstellen.
\[
A=\frac{2*3,3433\frac{m}{s^2}*Kg}{Cw*Roh*v^2}
\]
\[
A=\frac{2*3,3433\frac{m}{s^2}*Kg}{1,33*1,2\frac{Kg}{m^3}*15^2\frac{m^2}{s^2}} = 0,01862m^{2} \text{ oder } 186,2cm^{2}
\]
Eine Fläche von 186,2 cm² entspricht einem Durchmesser von ca. 16 cm.

\subsubsection{Bau}
Beim Bau der Fallschirme haben wir den Stoff von Regenschirmen verwendet. Dieser Stoff ist sehr geeignet, da er bereits in einer Art Halbkugelform mit acht aneinandergefügten Panels ist. Er bietet genug Widerstand, um den Belastungen des Fluges stand zu halten und hat als praktischen Nebeneffekt, das es regendicht ist. Um an den Stoff zu kommen muss das Gestell, des Regenschirmes, vorsichtig vom Stoff herunter geschnitten werden. Danach muss in der Mitte des Stoffes, wo alle acht Kanten aufeinandertreffen, ein rund 5 cm großes Loch geschnitten werden. Dort kann die gestauchte Luft leichter entweichen, statt am Rand unkontrolliert auszutreten. Würde sie dies nicht tun, so könnte es leicht zu gefährlichen Turbulenzen kommen. Nun kann der Fallschirm in die entsprechende Größe zugeschnitten werden. Am Rand des Fallschirms sollte ein zusätzlicher, ca. 1 cm breiter, Rand gelassen werden, der in den Fallschirm umgeklappt wird. Dieser muss mit einer Nähmaschine auf den Fallschirm festgenäht werden. Dies dient dazu, die Struktur des Randes besser zu schützen. Am Rand besteht die höchste Wahrscheinlichkeit, dass durch eine kleine Kerbe ein kompletter Riss entstehen könnte. Auf dieser Grundlage kann nun die Gaze, auf dem Rand, genäht werden. Dabei sollte beachtet werden, das beim Übergang von Fallschirm und Dose am Befestigungspunkt eine hohe Belastung auftritt. An diesem Punkt hat die Gaze einen gewaltigen Nachteil gegenüber der Schnüre. Hier kann es durch eine einmalige starke Belastung zum Riss kommen, der sich im Laufe des Fluges weiter ausbreiten kann. Daher haben wir uns dort entschieden eine Verstärkung mit einem sehr rissfestem Material ein zunähen, der die Spannung erst kompensiert und diese nach dem Entfalten des Fallschirms gut auf die Gaze verteilt.