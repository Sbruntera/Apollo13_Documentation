\section{Kurzbericht}
In den vergangenen Monaten haben wir uns damit beschäftigt, eine Vorabversion unseres Projektes fertigzustellen.\\ 
Im Bereich der Hardware sind bereits der Fallschirm, die Einrichtung unseres Mikrocontrollers, die eigentliche Dose, ein Teil der Sensorik und die eigentliche Sensorikplatine fertiggestellt. Als nächstes werden wir uns damit beschäftigen, das Innere der Dose, die Antenne, den Rest der Sensorik und die Integration aller Komponenten fertigzustellen und die gesamte Hardware ausgiebig zu testen. \\
Bei der Bodenstation ist bereits das Empfangen, das Verarbeiten, das Sichern, das Weiterleiten und das Visualisieren der Daten mithilfe einer dreidimensionalen Kartenvisualisierung, einem Graphen und diversen anderen Visualisierungen möglich. Die Benutzeroberfläche kann bereits beliebig angepasst und konfiguriert werden, während die Bodenstation in verschiedene Dateiformate exportieren und bereits mit beliebigen Satelliten und Übertragungsformaten genutzt werden kann. Als nächstes werden wir Fehler in der Bodenstation beheben, sie schöner gestalten, noch ein wenig nutzerfreundlicher machen und Features zur wissenschaftlichen Analyse der Daten hinzufügen. \\
Die Android-Applikation ist bereits dazu in der Lage, Daten über einen Hotspot von der Bodenstation zu beziehen und diese Daten in der Applikation mithilfe von Graphen darzustellen. Auch hier werden als nächstes Fehler behoben und weitere Features zur wissenschaftlichen Analyse der Daten hinzugefügt werden. \\
Im Anschluss an die Fertigstellung der Hardware werden wir natürlich auch das Gesamtsystem ausführlich testen. \\
Insgesamt müssen also bei der Hardware lediglich noch die letzten Teile fertiggestellt werden, während die Software nur noch einen letzten Schliff benötigt: Das Projekt schreitet gut voran.