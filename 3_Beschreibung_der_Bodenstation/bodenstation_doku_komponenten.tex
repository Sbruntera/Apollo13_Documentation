\subsubsection{Verwendete Komponenten}
Zum Erreichen unserer Ziele haben wir verschiedene Komponenten verwendet, welche einerseits der Datenvisualisierung und -analyse dienen, andererseits aber auch der Entkopplung und Skalierbarkeit für die Entwicklung.

Für die Bodenstation haben wir folgende Komponenten verwendet:
\begin{description}
	\item[\href{http://www.oracle.com/technetwork/java/javase/downloads/jdk8-downloads-2133151.html}{Java}] ist eine objektorientierte Programmiersprache. Diese wurde verwendet, da jedes unserer Gruppenmitglieder damit vertraut ist. Die Version Java 8 wurde verwendet um mächtige funktionale Features zu nutzen.
	\item[\href{https://netbeans.org/features/platform/}{Netbeans Platform}] das die Möglichkeit bietet, einfach eine integrierte, modulare und entkoppelte GUI-Applikation auf Basis von Java Swing zu entwickeln.
	\item[\href{http://junit.org/}{JUnit}] ist ein Framework, welches zum Erstellen von automatisierten Softwaretests dient.
	\item[\href{http://fazecast.github.io/jSerialComm/}{JSerialComm (zum Start des Projektes noch serial-comm)}] ist eine Bibliothek, welche das Auslesen serieller Schnittstellen ermöglicht 
	\item[\href{http://worldwind.arc.nasa.gov/java/}{NASA World Wind}] ist eine Software, welche Satelliten- und Luftbilder auf einem virtuellen Erdball darstellt. Daten der Bodenstation werden mittels dieser Software in Relation zur Höhe in Echtzeit visualisiert.
	\item[\href{http://jchart2d.sourceforge.net/}{JChart2D}] ist eine Grafik-Bibliothek, welche zur grafischen Visualisierung von Daten dient. Mithilfe dieser Bibliothek werden zweidimensionale Graphen erzeugt, welche empfangene Daten des Satelliten, in Relation zur Zeit oder anderen Daten, in einem Graphen darstellen.
	\item[\href{http://www.json.org/}{JSON (JavaScript Object Notation)}] ist ein Datenformat, welches zum Austausch von Daten zwischen Anwendungen angewandt wird. JSON ermöglicht es, Daten in Textform zu speichern und sie wieder zurück in ihre ursprüngliche Form zu interpretieren. Dieses Datenformat wird in der Bodenstationssoftware genutzt um Daten mit dem Satelliten auszutauschen, zu loggen, zu exportieren und zu importieren.
	\item[\href{http://git-scm.com/}{Git}] ist eine freie Versionsverwaltungssoftware. Diese Software verwendeten wir um die Versionen unserer Software auszutauschen und zu verwalten.
\end{description}