\section{Öffentlichkeitsarbeit}
\subsection{Website}
Unsere Website \href{www.team-gamma.de}{Team Gamma} wurde bereits für den europäischen CanSat Wettbewerb 2015 verwendet. Diese haben wir weiter geführt und dort in unregelmäßigen Abständen aktuelle Informationen über das Projekt veröffentlicht. Da die Website durch den europäischen Wettbewerb bei anderen europäischen Teams bekannt ist wird die Website in Englisch geführt. Man findet dort zusätzlich einige Dokumente, Fotos und Videos. Die Informationen auf der Website sind meist relativ detailliert verfasst. 

\subsection{Schülerzeitung}
In der Schülerzeitung unserer Schule sind bereits diverse Artikel über unser Projekt erschienen und sollen auch in Zukunft erscheinen. Diese Artikel handeln zumeist von dem Wettbewerb selber und gehen weniger auf die technischen Details ein.

\subsection{Präsentationen}
Da wir das CanSat Projekt bereits seit einiger betreiben kommt es immer wieder vor, dass wir es vor unserer Klasse präsentieren. Dies kommt zum Beispiel dann vor, wenn wir Teile des Projektes in Schulprojekte einfließen lassen. Zusätzlich haben wir, beispielsweise am Tag der offenen Tür unserer Schule, diversen Schulbesuchern das Projekt und den Wettbewerb näher gebracht.

\subsection{Ausstellung am MINT-Projekttag unserer Schule}
Im Schuljahr 2015/2016 findet an unserer Schule ein Tag der MINT Projekte statt. Dieser Tag wird von einer Schülergruppe unserer Parallelklasse organisiert und wir wollen an diesem Tag natürlich unser Projekt vorstellen.

\subsection{Logo}
Das Logo wurde ebenfalls aus Gründen der Wiedererkennbarkeit aus dem vorherigen Jahr übernommen. Das Aussehen des Logos wurde von drei Faktoren beeinflusst:
\begin{itemize}
	\item Das Zeichen in der Mitte soll dem Gamma Logo ähneln, welches zu unserem Teamnamen passt
	\item Das Zeichen soll zusätzlich, wenn man es um $180^\circ$ dreht, dem Lambda Logo ähneln. Da bei dem Entwurf unserer Antenne immer wieder auf Lambda gestoßen sind, sind daraus diverse interne Späße entstanden, die wir in das Logo einfließen lassen wollen.
	\item Das Logo des Computerspiel Halflife, welches von einigen Teammitgliedern gespielt wird
\end{itemize}