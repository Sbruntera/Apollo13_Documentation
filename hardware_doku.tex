\section{Der CanSat}
\subsection{Einleitung}
Wir haben uns für den Satelliten überlegt, dass dieser so weit wie möglich individuell sein sollte. Daher greifen wir nicht auf das, vom Wettbewerb bereitgestelte T-Minus CanSat Kit zurück. Stattdessen haben wir uns im Detail überlegt, welche Sensoren unseren Erwartungen entsprechen und wie wir diese bestmöglich innerhalb der Dose plazieren können. Zusätzlich möchten wir nicht auf eine Cola-Dose als Hülle zurück greifen, sondern möchten auch hier unser eigenes Design erschaffen.

\subsection {Hülle und Platzmanagement}
Wir haben uns dazu entschieden, die äußere Hülle aus GFK (Glasfaser verstärkter Kunststoff) zu fertigen. Dieses hat die Eigenschaften, dass er bei einem sehr geringen Gewicht, und bei einer geringen Wandstärke trotzdem eine gewisse Stabilität aufweißt. Aus dem GFK haben wir eine Röhre mit einem Innendurchmesser von 31,5 mm und einem Außendurchmesser von 33,5 mm laminiert. Diese Röhre wurde auf eine Länge von 111 mm gekürzt und gefeilt. Um die Röhre oben und unten zu verschließen haben wir uns bei Thyssen Krupp System Engineering zwei Aluminium Deckel fräsen lassen. Diese haben uns ebenfalls durch ihr geringes Gewicht überzeugt. \\
Um die Elektronik innerhalb der Hülle zu plazieren und zu befestigen haben wir uns dazu entschieden eine Zwischenwand mit einem 3D-Drucker anzufertigen. Diese Wand teilt die Hülle mittig und bietet so auf beiden Seiten Platz um unser Microcontroller Board und unsere Sensorik Platine zu befestigen. Beide Bauteile werden mittels vier Gewindestangen an der Wand befestigt. Durch die Technik des 3D-Druckens ist es möglich der Wand ein sehr geringes Gewicht bei einer verhältnismäßig hohen Stabilität zu verleihen. Zusätzlich gibt es uns die Möglichkeit die Wand millimetergenau zu gestalten. \\
Am unteren Ende der Wand befindet sich eine Aushölung, sowie ein Fuß. Diese ist zum einen dafür da um den Sharp Feinstaub Sensor zu befestigen. Zum anderen gibt der Fuß der Wand und somit dem gesamten Satelliten eine gewisse Stabilität. Der Fuß besitzt auf der einen Seite der Wand Bohrungen. Diese Bohrungen werden verwendet um die Aluminiumdeckel an der Wand zu befestigen. Da der Feinstaubsensor einen Luftzug benötigt befindet sich ein Durchlass innerhalb der Wand. Um das Microcontroller Board mit der Sensorik Platine zu verbinden existiert ein Fenster in der Mitte der Wand. \\
Um die Sensorik Platine und das Microcontroller Board an der Wand zu befestigen existieren vier Bohrungen.


