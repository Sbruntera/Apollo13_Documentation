\subsubsection{Realisierung der Desktopapplikation}
Während der Realisierung der Bodenstation kam es zu einigen Komplikationen. Zum einen wurde die Softwarearchitektur während der Implementationsphase geändert, so dass verschiedene Komponenten wie zum Beispiel Export- und Importkomponenten mehrfach realisiert wurden. Diese Architekturveränderung wurde vorgenommen und gewisse Komplikationen zu beseitigen, welche die modulare Strukturierung von Netbeans Plattform mit sich bringt. Wir starteten mit einer Architektur, welche jede wichtige Komponente als Modul benennt. Diese Architektur brachte zum einen das Problem, dass Abhängigkeiten zwischen Modulen nur in eine Richtung stattfinden kann. Die neue Architektur unterscheidet lediglich zwischen den Modulen API, GUI und Core. Die gesamte Architektur der Software zu ändern kostete uns viel Zeit.\\
Des Weiteren wurden verwendete Datentypen innerhalb der Software während der Implementationsphase geändert, sodass zusätzlich alle Komponenten, welche mit der Datenverarbeitung zu tun haben geändert werden mussten. Darunter fielen die gesamten Komponenten des Imports, Exports und der Live-Datenverarbeitung. Die Veränderung der benutzten Datentypen wurde von Long, String und Double zu einzig Double geändert, da alle Daten, welche von kompatiblem Satelliten gemessen werden in Double dargestellt werden können. Diese Änderung im Programmcode hat zwar einiges an Arbeit gekostet, doch der Umfang des Programmcodes wurde deutlich verringert und die Performance gesteigert.\\
Am Anfang der Realisierung der Software verlief die Versionsverwaltung mit Git nicht wie geplant. Nach fast jedem Versionsaustausch gab es Konflikte in unserem Projekt. Dies wurde hervorgerufen durch mangelnde Erfahrung mit Netbeans Plattform. Einzelne Dateien des Projektes werden ständig durch die IDE Netbeans verwendete, was uns nicht bekannt war. Dies wurde durch eine Vielzahl von Branches verkompliziert. Diese Konflikte zu lösen und das Problem endgültig zu beheben hat mehrere Stunden Zeit gekostet.