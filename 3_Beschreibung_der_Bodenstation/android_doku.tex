\subsubsection{Übersicht}
Die Hauptaufgaben der Android-App ist es, die Datenpakete, die von der Bodenstation über einen Hotspot gesendet werden, zu empfangen und live, in einem passenden Graphen, anzeigen zu können. Dabei soll Wert darauf gelegt werden, dass es möglich ist, alle Werte die gesendet werden, einzeln oder in Gruppen darzustellen. Dies soll ermöglichen, dass alle Daten verglichen werden können. Zusätzlich haben wir uns überlegt, die Werte auch in einem Balkendiagramm anzuzeigen. An dem Balkendiagramm soll die Differenz zwischen dem höchsten und dem niedrigsten gemessenen Wert berechnet werden. Diese Differenzen sollen dann in grün für positive Veränderungen und in rot für negative Veränderung dargestellt werden. Nebenher sollen ebenfalls einige Optionsmöglichkeiten vorhanden sein, um die Graphen nach den individuellen Wünschen des Nutzern zu gestalten.

\subsubsection{Plattform und Komponenten}
Für die Entwicklung der App wurde die IDE Android Studio verwendet. Durch eine übersichtliche Anordnung von Fenstern und einer geeignete Struktur ist diese eine ideale Programmierumgebung für die Android-App-Entwicklung. Android Studio ist speziell auf die Entwicklung für Android-Apps unter Java ausgelegt und bringt daher einige Features mit sich, welche die Entwicklung vereinfachen. Zusätzlich wurden zwei Libraries verwendet. Dabei handelt es sich um die androidplot-Library, welche es ermöglicht, vergleichsweise einfach Graphen zu erzeugen. Zusätzlich wurde eine JSON.Library verwendet, um den JSON String, welcher von der Bodenstation empfangen wird, zu parsen.
\subsubsection{Funktionen}
Für die App haben wir uns folgende Funktionen vorgenommen:
\begin{itemize}
	\item Anzeigen der Werte in einem Livegraphen
	\item Verwaltung der angezeigten Werte im Graphen während der Laufzeit
	\item Anzeigen von Differenzen von Werten im Balkendiagramm
	\item Manuelle Start/Stop Funktion für das Balkendiagramm
	\item Einstellung zur Geschwindigkeit des Graphen
	\item Einstellung zur Regelung der Anzahl der Werte, welche gleichzeitig angezeigt werden
	\item Die Möglichkeit, alle Funktionen mit einem Debug-Stream zu testen
\end{itemize}

\subsubsection{Nutzeranleitung}
Um die Team-Gamma-App nutzen zu können, benötigt der Nutzer unser apk-Datei. Diese wird online auf unserer Website im Download-Bereich verfügbar sein. Die App kann problemlos auf jedem Android-Smartphone, welches mit einer Androidversion von mindestens 2.3 läuft, installiert werden. Die App kann dann, wie jede andere App auch, gestartet werden. Vor dem Starten der App sollte der Nutzer sich mit dem Hotspot der Bodenstation verbinden um mögliche Komplikationen zu vermeiden. Nach dem Öffnen der App sollte nach kurzer Zeit das Menü erscheinen, das die App in drei Sektionen unterteilt: 

\begin{description}
	\item \textbf{Livegraph}
	Beim Antippen des Graphen-Icons erscheint ein leeres Koordinatensystem. Bei standardmäßigen Einstellungen werden zuerst keine Werte angezeigt. Nur falls sich das Gerät in der Nähe unseres zur Verfügung gestellten Hotspot befindet und genug Empfang hat, werden nach einigen Sekunden automatisch Werte von der Bodenstation live angezeigt. Diese werdem aber während der Vorbereitungsphase nur als gerade dargestellt. Durch das Drücken der Menütaste öffnet sich ein Fenster. Innerhalb dieses Fensters ist es möglich, die Werte auszuwählen, die angezeigt werden sollen.
	
	\item \textbf{Balkengraph}
	Im Balkengraph ist ebenfalls ohne den Debug-Stream nichts zu sehen. Steht die Verbindung mit der Bodenstation, so werden Werte im Graphen angezeigt. Falls die Berechnung nicht automatisch startet, kann man diesen auch manuell starten. Dazu muss der Menüknopf des Smartphones betätigt werden. In dem geöffneten Fenster kann der Graph gestartet und gestoppt werden. Der Graph wird dann den ersten Wert als ersten Punkt nehmen und den zuletzt gesendeten Punkt als zweiten Punkt nehmen. Die einzelnen Balken zeigen die Differenz zwischen dem ersten gemessenen Wert und dem aktuellsten Punkt an. Die Differenz wird dann in grün (positiv) oder rot (negativ) dargestellt. Beim Verlassen des Balkendiagramms wird dieses im Hintergrund weiter ausgeführt. Bei einer Marke von ungefähr null Metern stoppt der Graph die Berechnungen und zeigt die Werte solange an, bis sie nicht mehr benötigt werden. Beim Beenden des Programm werden diese unwiderruflich gelöscht.
	
	\item \textbf{Optionen}
	Die Optionen in der App geben dem Nutzer die Möglichkeit, die Geschwindigkeit des Livegraphen zu bestimmen. Außerdem kann dort eingestellt werden, wie viele Werte gleichzeitig angezeigt werden sollen. Für das Testen und Anzeigen von Werten gibt es dort die Möglichkeit, einen Debug-Stream einzuschalten/auszuschalten.
\end{description}