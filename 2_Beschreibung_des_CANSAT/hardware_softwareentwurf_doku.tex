\subsection{Softwaredesign}
\subsubsection{Python als Programmiersprache}
Als Programmiersprache zur Programmierung des Beaglebone Black’s, haben wir uns für die Programmiersprache Python entschieden. Es wäre zwar ebenfalls möglich gewesen den Mikrocontroller mit den Sprachen JavaScript, Java, C, C++, C# und vielen weiteren Sprachen zu programmieren, da es sich bei dem Beaglebone um ein embedded-System handelt, welches praktisch alle Programmiersprachen unterstützt, sofern entsprechende Librarys existieren. Allerdings haben wir uns aufgrund der Tatsache, dass Python im Gegensatz zu Java nicht objektorientiert geschrieben werden muss, und wir auf der Hardwareseite möglichst auf objektorientierte Programmierung verzichten wollen, da sie nicht nötig ist, für Python entschieden. Ein weiteres wichtiges Argument war die gute Python-Library, welche von einer großen Community permanent gewartet und aktualisiert wird.
\subsubsection{Datenverarbeitung auf dem Beaglebone}
Die Datenverarbeitung auf dem Beaglebone verläuft relativ simpel. Zunächst werden alle von den Sensoren aufgezeichneten Daten, bei Sensoren mit I2C Anbindung mithilfe von Librarys und bei den anderen mithilfe von Umrechnungsalgorithmen, gesammelt. Anschließend werden alle gesammelten Daten in einen JSON-String geparsed, welcher mithilfe unserer Antenne an die Bodenstation übermittelt wird. Diese übernimmt die weitere Verarbeitung und Darstellung der Messdaten. 