\section{Projektplanung}
\subsection{Zeitplan der CanSat Vorbereitung}
Die Zeitplanung ist ausgerichtet für den Zeitpunkt der Abgabe unseres P5, da wir uns gewünscht haben, zu diesem Zeitpunkt mit dem Projekt fertig zu sein. Dieser Zeitplan wurde jedoch von Anfang an sehr kritisch gesehen. Daher ist es nicht verwunderlich, dass der Fortschritt des Projektes geringer ist, als er zum jetzigen Zeitpunkt eigentlich seien sollte. Dies ist jedoch nicht dramatisch, da bis zum Wettbewerb genügend Zeit ist die restlichen Arbeitspakete abzuarbeiten. Das gesamte Management der Arbeitspakete und des Zeitaufwandes wurde mit der Projektmanagementsoftware \href {www.redmine.org} {Redmine} erledigt. Da diese auf unserem Server unter \href{http://redmine.gamma-team.de}{redmine.gamma-team.de} erreichbar ist kann jedes Teammitglied zu jedem Zeitpunkt den Fortschritt der Arbeit verfolgen. Die Planung der beiden Halbgruppen ist größtenteils voneinander getrennt. Es gibt jedoch gemeinsame Meilensteine, welche von beiden Gruppen eingehalten werden sollen. Bevor die Arbeit der Halbgruppen begonnen hat gab es eine allgemeine Projektfindungsphase. In dieser Phase wurde ein grober Zeitplan festgelegt und es wurden alle relevanten Systeme (Webserver, Projektmanagementsoftware, GitLab etc.) aufgesetzt und eingerichtet um später einen reibungslosen Ablauf der Arbeitsphase zu garantieren. Die Idee und die Spezialisierung der Idee für das gesamte Projekt entstand ebenfalls in dieser Zeit. Anschließen wurde eine separate Zeitplanung in den beiden Halbgruppen erstellt, welche im Nachfolgenden erläutert wird.
\subsubsection{Zeitplan der Hardware Gruppe}
Innerhalb der Hardwaregruppe wurden versucht die meisten Aufgaben zu parallelisieren. Jedes Teammitglied hat sein eigenes spezielles Aufgabengebiet. jedoch herrscht trotzdem ein stetiger Austausch zwischen den Teammitgliedern. Grund für die Parallelisierung war, dass in unseren Augen die meisten Aufgaben  nur die Aufmerksamkeit einer Person benötigen. Es ist nur selten erforderlich, dass mehrere Teammitglieder an ein und dem selben Arbeitspaket arbeiten. Der gesamte Arbeitsprozess wurde in diverse Abschnitte gegliedert. Diese Abschnitte lassen sich auch im GANTT Diagramm im Anhang dieses Dokumentes wiederfinden. Bei den Abschnitten handelt es sich um folgende:
\begin{itemize}
\item Planung: Erstellung von Arbeitspaketen, sowie eine Verteilung dieser und eine Erstellung diverser Diagramme
\item Fallschirm: Gestaltung und Bau des Bergungssystems.
\item Sensorik: Dieser Abschnitt behandelt das Heraussuchen, Bestellen und Testen passender Sensoren für unser Projekt.
\item Beagleboard: Festlegung der Programmiersprache, IDE und der Recherche zu den elektrotechnischen Eigenschaften des Boards
\item Dose: Design und Bau der Hülle und der Deckel der Dose
\item Dosenmanagement: Design und Bau des inneren der Dose, sowie die Integration der Sensoren in das Gesamtsystem
\end{itemize}

Die einzelnen Abschnitte sind in diverse Arbeitspakete unterteilt, Personen zugewiesen und mit einem Zeitraum versehen.

\subsubsection{Zeitplan der Software Gruppe}
In der Softwaregruppe haben wir uns wie in der Hardwaregruppe dafür entschieden, die Aufgeben untereinander zu verteilen. Dabei haben wir zu erst die Bodenstation und die App komplett von einander getrennt. Die Bodenstation ... (INSERT BODENSATION)

Die App hingegen haben wir erst, im sehr spät angefangen dran zu arbeiten. So konnten wir diese in den letzten Monaten in "Wochenarbeit" unterteilt. Mit dieser Methode konnten wir in jeweils 2 Woche kontrolliert an einem anderen Komponente arbeiten. Falls solch ein Komponent zu lange Zeit braucht, war es möglich am Wochenende dran zu arbeiten oder diesen in zwei Arbeitspakete zu unterteilen. Zusammen gefasst waren es insgesamt 7 Arbeitspakete:
\begin{itemize}
	\item Der Debugger, der die Livedaten simulieren sollte
	\item Liniengraph GUI/LOGIC
	\item Balkendiagramm GUI/LOGIC
	\item Optionen
	\item Menü
\end{itemize}

\subsection{Einschätzung der Mittel}
\subsubsection{Budget}
% \begin{tabular}{p{1,5cm}p{1,5cm}p{3,5cm}p{6,5cm}rrrl}
\label{subsubsec:Budget}

Um das CanSat Projekt zu finanzieren konnten wir aktuell noch keine Sponsoren finden. Jedoch konnten wir uns mit unserem Schulverein verständigen, welcher uns finanziell unterstützen wird. Da wir nicht auf das T-Minus Kitt zurückgreifen sondern stattdessen ein anderes Mikrocontroller Board verwenden können wir ungefähr 150  \euro  sparen. Der 200 \euro Watterot Gutschein, welcher vom Wettbewerb gestellt wird, ist in unseren Rechnung noch nicht inbegriffen. Dies liegt daran, dass noch nichts bei Watterot bestellt wurde, bzw. die Bestellung lange vor der Annahme am Wettbewerb getätigt wurde.
Im Nachfolgenden sind alle Ausgaben und Einnahmen aufgelistet.
\begin{table}[htbp]
  \centering
    \begin{tabular}{p{1,7cm}p{1,5cm}p{3,5cm}p{6,5cm}rrrl}
    \toprule
    \textbf{Ausgabe} & \textbf{Datum} & \textbf{Empfänger} & \textbf{Grund} \\
    \midrule
    -12,16 \euro  & 08.01.2015 & Watterott & BMP180 Breakout \\
    -28,99 \euro  & 09.01.2015 & eBay - rcskymodel & Ultimate GPS \\
    -14,32 \euro  & 10.01.2015 & Spark Fun Electronics & UV-Sensor \\
    -51,99 \euro  & 10.01.2015 & Amazon & Beagle Bone Black \\
    -17,30 \euro  & 01.12.2014 & eBay - hdt-preiswert & 
GFK-Set 1kg Polyesterharz + 20g Härter + $2m^2$ Glasfasermatte \\
    -3,54 \euro  & 23.03.2015 & toom baumarkt & 6 x Schleifpapier \\
    -3,79 \euro  & 23.03.2015 & toom baumarkt & Filzrolle \\
    -4,49 \euro  & 23.03.2015 & toom baumarkt & Plüschwalzen \\
    -2,19 \euro  & 23.03.2015 & toom baumarkt & Mundschutz \\
    -1,99 \euro  & 23.03.2015 & toom baumarkt & Farbwanne \\
    -4,99 \euro  & 23.03.2015 & toom baumarkt & Einmalhandschuhe \\
    \bottomrule
    - 145,75 \euro & & & \\
    \bottomrule
    \end{tabular}%
    \caption{Ausgaben}
  \label{tab:budgetausgaben}%
\end{table}%

\begin{table}[htbp]
  \centering
    \begin{tabular}{p{1,7cm}p{1,5cm}p{3,5cm}p{6,5cm}rrrl}
    \toprule
    \multicolumn{1}{c}{\textbf{Einnahmen}} & \textbf{Datum} & \textbf{Absender} & \textbf{Grund} \\
    \midrule
              17,30 \euro  & 01.12.2014 & Alexander Brennecke & GFK-Kauf \\
           107,46 \euro  & 10.01.2015 & Alexander Brennecke & Sensorenkauf \\
              20,99 \euro  & 23.03.2015 & Alexander Brennecke & toom Einkauf \\
    \bottomrule
    145,75 \euro & & & \\
    \bottomrule
    \end{tabular}%
	\caption{Einnahmen}
  \label{tab:budgeteinnahmen}%
\end{table}%

\subsubsection{Externe Unterstützung}
Externe Unterstützung erhielten wir von vielen Lehrern unserer Schule, welche uns Fragen zur Elektrotechnik und Softwareprogrmmierung beantworten konnten. Zusätzlich haben wir finanzielle Unterstützung durch den Schulverein unserer Schule erhalten (siehe~\ref{subsubsec:Budget}).
Unterstützung außerhalb unserer Schule erhileten wir durch folgende Personen/Organisationen:

\begin{itemize}
	\item Das \href{https://www.hackerspace-bremen.de/}{Hackerspace Bremen e.V.}, welches uns ihren 3D-Drucker zur Verfügung gestellt hat. Zusätzlich konnten wir dort unsere Platine ätzen.
	\item \href{http://de.wikipedia.org/wiki/Martin_Schneider_(Nachrichtentechniker)} {Prof. Martin Schneider} von von dem Hochfrequenzlabor der Universität Bremen, welcher uns geholfen hat unsere Antenne an die Frequenz und die Wellenimpedanz anzupassen.
	\item  Das Umweltlabor der \href{http://www.atlas-elektronik.com/atlas-elektronik/}{Atlas Elektronik GmbH} hat uns geholfen den CanSat, hinsichtlich seiner Stabilität, zu testen und die Sensoren korrekt zu kalibrieren.
\end{itemize}