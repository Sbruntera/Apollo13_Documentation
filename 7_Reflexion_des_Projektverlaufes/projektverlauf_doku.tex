\section{Reflexion des Projektverlaufes}
\subsection{Reflexion der Hardwaregruppe}
Als wir angefangen haben das gesamte Projekt zu planen haben wir uns als Ziel gesetzt Ende Mai fertig zu sein. Dieses Datum haben wir Aufgrund der Abgabe unseres P5 gewählt, für welches wir das CanSat Projekt ebenfalls einreichen wollen. Uns war bewusst, dass dies ein sehr hoch gestecktes Ziel ist. Im Nachhinein haben wir relativ schnell gemerkt, dass wir dieses Ziel nicht erreichen können. Diese Verzögerung wurde durch mehrere Faktoren hervorgerufen. Dazu zählt der enorm hohe Anspruch den wir uns selber gesetzt haben. Dieser hatte immer wieder zur Folge, dass viele Dinge mehrfach oder gründlicher gemacht werden mussten, als es zu Anfang geplant war. Zum anderen haben wir verhältnismäßig lange gebraucht um uns auf eine finale Idee festzulegen und diese zu präzisieren. Da wir uns jedoch kontinuierlich zum arbeiten getroffen haben konnten wir dennoch gute Fortschritte erzielen. Wir lagen zwar die meiste Zeit über hinter unserem Zeitplan, konnten jedoch die Reihenfolge der zu bearbeitenden Aufgabenpakete größtenteils einhalten.

\subsection{Reflexion der Softwaregruppe}
\subsubsection{Desktopapplikation}
Rückblickend denken wir, dass die Bodenstation nicht unseren Anforderungen entspricht. Die Bodenstation ist insgesamt nicht komplett fertig geworden. Die Endversion, welche wir zur Deadline vorzeigen können, stellt lediglich einen Snapshot der fertigen Bodenstation dar: Verschiedene grafische Anbindungen fehlen noch und die Benutzeroberfläche ist generell noch unschön. Alle Funktionalitäten sind bereits in der Bodenstation implementiert, jedoch fehlt bei einigen Funktionalitäten noch die grafische Anbindung. Dazu, dass wir uns nicht komplett an unsere Planung halten konnten, haben verschiedene Faktoren geführt:

Wir sind insgesamt zu experimentierfreudig an das Projekt herangegangen. Da wir innerhalb des Projektes die für uns komplett neue Technologie ``Netbeans Platform'' genutzt haben, haben wir am Anfang des Projektes zu viel darauf gesetzt, dass wir mit Netbeans Platform gut zurecht kommen würden. Dem war im Endeffekt nicht so: Wir mussten feststellen, dass die von Netbeans Platform verwendete Architektur uns lediglich in allen möglichen Bereichen limitierte, anstatt uns wirklich Arbeit abzunehmen. Die von Netbeans Platform vorgeschlagene Modulkommunikation, über so genannte Lookups, machte es unmöglich, bestimmte Komponenten so anzusprechen, wie wir es ursprünglich geplant hatten. Diese Art der Modulkommunikation kostete uns insgesamt viel zu viel Zeit. Zudem kam noch, dass Netbeans Platform sich nicht effektiv zusammen mit Git verwenden lies: Die von Netbeans Platform generierten Konfigurationsdateien sorgten immer wieder für Merge-Konflikte und andere Probleme. Zudem kam auch noch, dass der Netbeans-Cache immer wieder in falsche Zustände geriet, was ebenfalls zu Problemen bei der Dateiverwaltung mithilfe von Git geführt hat, bis wir endgültig einen Skript geschrieben haben, welcher den kompletten Netbeans-Cache zurücksetzt. Der einzige Bereich, wo uns Netbeans Platform Zeit gespart hat, war die dynamische Benutzeroberfläche, welche wir dank Netbeans Platform nicht mehr entwickeln mussten. Insgesamt hat uns Netbeans Platform sehr viel Zeit gekostet, und wir sind der Meinung, dass wir das Projekt ohne jegliche Probleme hätten rechtzeitig fertigstellen können, wenn wir nicht auf endlose Probleme mit Netbeans Platform gestoßen wären.

Zusätzlich dazu, dass wir unterschätzt haben, wie viel Aufwand wir in Netbeans Platform stecken müssten, kommt auch noch, dass wir den Aufwand zur Entwicklung der GUI-Anbindungen unterschätzt haben. Wir sind ursprünglich davon ausgegangen, dass die grafischen Anbindungen der jeweiligen Features immer zum jeweiligen Meilstein passend abgeschlossen würden. Dem war schließlich nicht so: Die eigentlichen Funktionalitäten wurden immer weit vor den grafischen Anbindungen fertiggestellt. Dies ging sogar so weit, dass die ersten grafischen Anbindungen der Basisversion fertiggestellt wurden, als bereits Teilfunktionalitäten für die finale Version fertiggestellt waren. Dies lag ebenfalls teilweise daran, dass wir, damit die jeweiligen Funktionalitäten eine grafische Anbindung erhalten konnten, erst die Modulkommunikation von Netbeans Platform zum Laufen bekommen mussten, was eine gigantische Blockade im Fortschritt des Projektes darstellte.

Im Endeffekt haben sich einige Gruppenmitglieder auch mit ihrer Verantwortung und ihren Aufgaben ein wenig überschätzt, was ebenfalls zu Verzögerungen geführt hat. Die Aufteilung zwischen den Gruppenmitgliedern war also nicht perfekt balanciert, was auch zu Problemen und Verlangsamungen geführt hat.

Die Kommunikation zwischen den Teammitgliedern, welche an der Bodenstation gearbeitet haben, hat insgesamt einigermaßen gut funktioniert. Es war immer allen Teammitgliedern klar, was getan werden musste, und generall war auch nie ein Teammitglied im Leerlauf, ohne eine Aufgabe zu haben.

Insgesamt hat die Bodenstation einige interessante Features, erscheint jedoch unfertig. Für die Nutzung mit unserem eigenen Satelliten ist die Bodenstation wohl ausreichend, obwohl einige ursprüngliche Features nach wie vor keine richtige grafische Anbindung haben.

\subsubsection{Android Applikation}
Zusammengefasst war das Programmieren der Androide App am Anfang ein schwieriges unterfangen. Gleich zu Beginn der Entwicklungsphase, ist es schwer geworden, im User-Interface auf einzelne Komponenten der GUI zu zugreifen. Dabei sind wir auch gleich auf das bisweilen größte Problem gestoßen. Der Debugger-Thread hat durch schlechtere Strukturen wären der Laufzeit auf eine Liste zu gegriffen, die auch vom Aktionlisener benutzt wird. Es war eine fünfzig/fünfzig Chance, das die Werte im Livegraph gewechselt worden sind oder die Applikation abstürzt. Dies war ein herber Rückschlag, da die Struktur dadurch umgeschrieben werden musste. Zurzeit ist das Problem besser unter Kontrolle, aber noch nicht ganz von der Welt. Im weiteren Verlauf gab es nur noch wenige Fehler die zu beheben waren, daher hat es doch viel Spaß gemacht, weil jede neue Methode oder Zeile im Code fast immer optisch auf dem Display zu erkenne war. 

\subsection{Reflexion der Zusammenarbeit zwischen den Teams}
Da der inhaltliche Schwerpunkt der beiden Teams relativ wenig miteinander zu tun hat sollte es theoretisch relativ wenige Berührungspunkte geben. Dies war bei unserer Projektarbeit jedoch nicht so. Da die Arbeit der beiden Halbgruppen zur gleichen Zeit in der gleichen Räumlichkeit stattfand war es oft so, dass teamübergreifend  diskutiert wurde. Dies hat den Vorteil, dass beide Teams nochmal einen anderen Blick auf eventuelle Problemstellungen bekommen und so einfache oder bessere Lösungen für Probleme finden können. Zusätzlich lief die Absprache über den Datenaustausch zwischen Bodenstation und CanSat sehr gut. Die beiden Teams haben also hervorragend kooperiert und gemeinsam versucht ein bestmögliches Gesamtprodukt zu erschaffen.
