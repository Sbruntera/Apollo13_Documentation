\section{Einleitung}
\subsection{Die Idee}

Die Idee hinter dem gesamten Projekts bezieht sich auf die extremen Umweltbelastung und ihre Folgen für den Menschlichen Körper. Ausschlaggebend für diese Idee ist ein Zeitungsartikel der Zeit, welcher über eine drohende Klage der EU-Kommission in Brüssel berichtet. (vgl. Die Zeit, 24.10.2014). Die EU-Kommission droht mit einer Klage gegen Deutschland, da die deutsche Bundesregierung bisher zu wenig Aufwand betreibt, um die Feinstaubkonzentration in der Luft zu reduzieren. Wir möchten diesen Aspekt aufgreifen und Messungen druchführen um die tatsächlichen Werte zu bestimen. Der CanSat Wettbewerb eignet sich optimal dazu, da er uns die Möglichkeit bietet die Messungen nicht nur auf dem Boden sondern in verschiedenen Schichten der Athmospähre durchzuführen. Feinstäube stehen in Verdacht, Krankheiten wie Asthma, Herz-Kreislauf Beschwerden und Krebs zu begünstigen.

Da der menschliche Körper nicht nur durch Feinstaub belastet wird haben wir uns entschlossen auch die Intensität der UV-Strahlung, welche die Hauptursache für Hautkrebserkrankungen ist, zu messen. Zusätzlich soll auch der Ozonwert bestimmt werden, da Ozon bereits in geringen Konzentrationen gesundheitsschädlich ist und zu Reizungen der Atemwege führen kann.

Für sich genommen ist jede dieser drei Größen schädlich für den Menschen. Im Zuge des Projektes wollen wir jedoch versuchen heruaszufinden, ob es einen Zusammenhang zwischen ihnen gibt. Beispielsweiße ist herauszufinden, ob ein höherer Ozon Gehalt gleichzeitig einen niedrigereren Feinstaubgehalt mit sich bringt.

Zusätzlich zum Bau des Messystems im CanSat ist es unser Ziel eine einwandfreie Verarbeitung, Analyse und Präsentation der gemessenen Werte zu erzielen. Um dies zu garantieren programmieren wir ein eigenes Analysetool. Dieses Tool ermöglicht es uns die gemessenen Werte, während des Fluges des Sateliten, auszuwerten. Die Werte sollen dabei anschaulich und in Abhängigkeit zueinander dagestellt werden.

Um die Daten auch mobil verfügbar zu haben wollen wir eine Android Applikation bereitstellen. Diese Applikation soll vorerst nur für unser Projekt optimiert sein, bei Erfolg jedoch auch die Werte andere Teams anzeigen können.

\subsection{Das Team}

Das gesamte Team besteht aus sieben Schülern und zwei betreuenden Lehrern. Die sieben Schüler sind jedoch intern in mehrere kleinere Teams aufgeteilt. Innerhalb der Teams ist jedoch kein Teammitglied vollkommen an seine Aufgaben gebunden, da uns ein guter Austausch und eine hervorragende Zusammenarbeit zwischen den einzeln Teammitgliedern und Teams wichtig ist. Die Arbeit der Gruppen und der einzelnen Personen werden im folgenden erläutert:

Das Hardware Team besteht aus drei Personen, welche sich um den Bau des Sateliten selber, dem Design und dem Bau der Dose sowie der Programmierung des Mikrocontrollers kümmern. Zu disem Team zählen folgende Personen:

Alexander Brennecke ist verantwortlich für das Design der Dose. Dazu zählt die Konstruktion der eigentlichen Dose und die Anordnung der Sensoren im inneren der Dose.

Till Schlechtweg ist verantwortlich für die Funktionalität des Mikrocontrollers und den ausgewählten Sensoren.

Steffen Wißmann ist verantwortlich für die Übertragung der Daten zur Bodenstation und dem Programmcode des Mikrocontrollers.

Das Software Team besteht aus vier Personen, welche sich um das Programmieren des Analysetools und der Android Applikation kümmern. Dieses Team besteht aus folgenden Personen:

Robin Bley

Alexander Feldmann

Marc Huisinga

Kevin Neumeyer

Zudem gib es ein Team, bestehend aus Alexander Brennecke und Till Schlechtweg, zur Organisation, Kommunikation mit Sponsoren und Öffentlichkeitsarbeit.






