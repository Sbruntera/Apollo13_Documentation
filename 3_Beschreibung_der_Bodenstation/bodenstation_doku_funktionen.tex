\subsubsection{Funktionen}
\paragraph{Nutzerfreundlichkeit}
Die Bodenstation wurde so entwickelt, dass der Nutzer der Bodenstation sich nicht um Implementationsdetails scheren muss und die Bodenstation als zentralen Empfänger für Daten von seinem Satelliten nutzen kann, ohne dabei etwas anderes als die grafische Benutzeroberfläche zu verwenden. 

Ein wichtiger Faktor im Bereich der Nutzerfreundlichkeit ist die dynamische Benutzeroberfläche, welche es dem Nutzer erlaubt, verschiedene GUI-Komponenten in Teilpanels innerhalb der Applikation anzuzeigen und umzustellen. Diese Dynamik ermöglicht es dem Nutzer, die Benutzeroberfläche, welche für die Analyse seiner Daten am Besten ist, mithilfe der verschiedenen Panels einzustellen.

Ein weiterer, wichtiger Faktor ist, dass der Nutzer die Bodenstation so anpassen kann, wie es für die Datenübertragung seines Satelliten am Besten ist. Alle Teilkomponenten der Datenübertragung sind über die grafische Benutzeroberfläche austauschbar, was es sehr einfach macht, die Bodenstation auf die Datenübertragung des jeweiligen Satelliten anzupassen.

Alle Visualisierungskomponenten sind zudem intuitiv aufgebaut, sodass es nicht kompliziert ist, sich seine Daten mithilfe der Visualisierungskomponenten anzusehen. Die Anzeige über den 3D-Globus erlaubt es beispielsweise, den Globus beliebig zu bewegen und die Position auf dem Globus zu verändern, während der Graph es erlaubt, dass die Axen des Graphen beliebig ausgetauscht werden können.

\paragraph{Erweiterbarkeit}
Bei der Entwicklung der Bodenstation haben wir darauf geachtet, dass die Bodenstation auf einer skalierbaren Architektur aufgebaut ist. Dies ermöglicht es, leicht neue Module und Funktionalitäten zur Bodenstation hinzuzufügen, ohne dabei besonders viel Code abzuändern. Auf die folgenden Weisen ist die Bodenstation skalierbar:
\begin{itemize}
	\item Neue GUI-Komponenten können sehr leicht hinzugefügt werden. Das Erstellen und Einbinden eines GUI-Komponenten umfasst lediglich die Erstellung eines neuen Netbeans-TopComponents.
	\item Unterschiedliche Satelliten können ohne eine erneute Kompilierung hinzugefügt und verändert werden, indem man die Konfigurationsdateien der Bodenstation anpasst, welche zur Laufzeit von der Bodenstation geladen werden.
	\item Neue Konfigurationsformate können leicht hinzugefügt werden, indem man die neue Konfigurationsimplementation unter einem Interface in der Applikation hinzufügt.
	\item Es ist ohne viel Aufwand möglich, die verschiedenen Datenquellen, aus denen Daten bezogen werden, auszutauschen. Möchte man also die Daten von einer anderen Quelle als einem USB-Port beziehen, so ist dies leicht zu implementieren, indem man ledigliche eine neue DataSource implementiert und zur Applikation hinzufügt.
	\item Verschiedene Datenübertragungsformate können ebenfalls ohne viel Aufwand hinzugefügt werden, indem neue Formate unter dem DataFormat-Interface implementiert werden. Die Applikation ist also nicht auf JSON als Übertragungsformat limitiert.
	\item Die verschiedenen Datenempfänger können auch leicht angepasst werden, indem man neue Datenempfänger unter dem Receiver-Interface implementiert, wodurch die verschiedenen Logging-Formate erweitert werden können.
	\item Daten können aus beliebigen Dateiformaten importiert werden, was ebenfalls leicht erweiterbar ist, indem man neue Import-Dateiformate über das Importer-Interface implementiert.
	\item Export-Formate können leicht erweitert werden, indem man neue Export-Formate unter dem Exporter-Interface implementiert.
\end{itemize}

\paragraph{Features}
Da die Software der Bodenstation auf dem Framework Netbeans Plantform basiert, lassen sich einzelne graphische Module kombinieren, welche sich per drag and drop verschieben lassen. Die Größe und Position dieser Module und des gesamten Frames lassen sich beliebig verändern. Des weiteren bietet die Software die Möglichkeit, Daten von verschiedenen Satelliten zu empfangen. Empfangene Daten lassen sich mittels der graphischen Oberfläche in Graphen anzeigen, welche sich verschieden kombinieren lassen. Außerdem lassen sich die empfangenen Daten zwischenspeichern und anschließend in verschiedene Dateiformate exportieren oder live in einer Datei loggen. Diese exportierten oder geloggten Daten lassen sich anschließend wieder einlesen und anzeigen. CSV, TXT, JSON, KML und PNG sind Dateiformate, welche exportierbar sind. Davon lassen sich exportierte CSV- und JSON Dateien wieder einlesen und visualisieren. TXT-Dateien werden formatiert und somit gut leserlich für den Nutzer exportiert, während exportierte KML-Dateien per Google Earth geöffnet und graphisch visualisiert werden können. Ein weiteres Feature der Bodenstationsoftware ist die Datenvisualisierung per Nasa World Wind als Modul in der graphischen Oberfläche. Diese Visualisierung zeigt den Flug des Satelliten auf einem virtuellen Globus, mittels Satelliten- und Luftbilder, und zeigt auf jeder gemessenen GPS-Koordinate die gemessenen Werte der Sensoren des Satelliten. Unter anderem bietet dieses Modul der Software die Möglichkeit, an die virtuelle Erdkugel heranzuzoomen und einzelne Elemente dreidimensional darzustellen. Darstellungen mittels dieses virtuellen Erdballs sind sowohl in Echtzeit mittels eines Streams vom Satelliten als auch als Import aus einer Datei möglich.
