\section{Beschreibung des CanSat}


% Blockdiagramm (irgendjemand)
\subsection{Missionsüberblick}
Wir haben uns für den Satelliten überlegt, dass dieser so individuell wie möglich sein soll. Daher greifen wir nicht auf das vom Wettbewerb bereitgestellte T-Minus-CanSat-Kit zurück. Stattdessen haben wir uns im Detail überlegt, welche Sensoren unseren Erwartungen entsprechen und wie wir diese bestmöglich innerhalb der Dose platzieren können. Zusätzlich möchten wir nicht auf eine standardisierte Getränkedose als Hülle zurückgreifen, sondern auch hier unser eigenes Design erschaffen.

% 3D Skizze, Erklärung der einzelnen Bestandteile (Alexander B.)
\subsection{Mechanisches und Strukturdesign}

Wir haben den CanSat in drei Komponenten aufgeteilt: Die Hülle, die Innenwand und die Sensorik Platine. Diese drei Komponenten bilden den Hauptbestandteil des CanSats und haben maßgeblich zu dem mechanischen und strukturellem Design beigetragen. Im nachfolgenden wird kurz auf jeden dieser Komponenten eingegangen und die exakte Funktion im Zusammenhang erklärt.

\subsubsection{Die Hülle}
Wir haben uns dazu entschieden, die äußere Hülle aus GFK (Glasfaser verstärkter Kunststoff) zu fertigen. Dieses hat die Eigenschaften, dass er bei einem sehr geringen Gewicht, und bei einer geringen Wandstärke trotzdem eine gewisse Stabilität aufweißt. Aus dem GFK haben wir eine Röhre mit einem Innendurchmesser von 31,5 mm und einem Außendurchmesser von 33,5 mm laminiert. Diese Röhre wurde auf eine Länge von 111 mm gekürzt und gefeilt. Um die Röhre oben und unten zu verschließen haben wir uns bei Thyssen Krupp System Engineering zwei Aluminium Deckel fräsen lassen. Diese haben uns ebenfalls durch ihr geringes Gewicht und ihre hohe Stabilität überzeugt.

\subsubsection{Innenwand}
Um die Elektronik innerhalb der Hülle zu platzieren und zu befestigen haben wir uns dazu entschieden eine Wand anzufertigen. Diese Wand teilt die Hülle mittig und bietet so auf beiden Seiten Platz um unser Mikrocontroller Board und unsere Sensorik Platine zu befestigen. Beide Bauteile werden mittels vier Gewindestangen an der Wand befestigt. Durch die Technik des 3D-Druckens ist es möglich der Wand ein sehr geringes Gewicht bei einer verhältnismäßig hohen Stabilität zu verleihen. Zusätzlich gibt es uns die Möglichkeit die Wand millimetergenau zu gestalten. \\
Am unteren Ende der Wand befindet sich eine Aushöhlung, sowie ein Fuß. Diese ist zum einen dafür da um den Sharp Feinstaub Sensor zu befestigen. Zum anderen gibt der Fuß der Wand und somit dem gesamten Satelliten eine gewisse Stabilität. Der Fuß besitzt auf der einen Seite der Wand Bohrungen. Diese Bohrungen werden verwendet um die Aluminiumdeckel an der Wand zu befestigen. An der oberen Seite der Wand befinden sich ebenfalls solche Bohrungen um den oberen Deckel der Hülle zu befestigen. Da der Feinstaubsensor einen Luftzug benötigt befindet sich ein Durchlass innerhalb der Wand. Um das Mikrocontroller Board mit der Sensorik Platine zu verbinden existiert ein Fenster in der Mitte der Wand. Um die Sensorik Platine und das Mikrocontroller Board an der Wand zu befestigen existieren vier Bohrungen.

\subsubsection{Die Sensorik Platine}
Die Sensorik Platine ist eine von uns geätzte Platine, welche mit unseren Sensoren bestückt ist. Es gibt mehrere positive Aspekte, die eine eigene Platine mit sich bringt. Zum einen bietet sie eine stabile Plattform für die Befestigung der Sensoren. Zum anderen sparen wir uns dadurch eine Menge Kabel, welche deutlich störanfälliger sind als eine Platine. Die Platine hat an den entsprechenden Stellen Bohrungen um sie mit der Zwischenwand und dem Mikrokontrollboard zu verbinden. Die Platine bietet Platz für folgende Module:

\begin{itemize}
	\item BMP108 Drucksensor: Misst den Luftdruck und gibt diesen, sowie die daraus berechnete Höhe zurück
	\item Sparkfun UV Sensor: Misst die Intensität des Spektrums 270-380 nm, welches dem UVA und UVB Spektrum entspricht
	\item TMP006 Infrarot Temperatursensor: Misst die Temperatur eines dünnen Aluminiumstückes in der Außenwand 
	\item Adafruit Ultimate GPS: Bestimmt die aktuelle Position sowie die Höhe
	\item APC220 Transceiver Modul: Sendet die Daten als JSON String zur Bodenstation
	\item Steckplatz zum Anschluss des Sharp Feinstaub Sensors: Misst den Anteil der Partikel, welche kleiner als 10 \textmu m sind
	\item Steckplatz zum Anschluss an das Mikrokontrollerdboard: Bildet die Schnittstelle zwischen BeagleBone und Senorik Platine
\end{itemize}

\subsubsection {Fachliche Grundlage}
Um die 3D gedruckte Wand zu erzeugen wurde die 3D Moddelierungssoftware \href{http://www.sketchup.com/de} {Sketchup} von Google verwendet. Sketchup bietet die Möglichkeit vergleichsweiße einfach 3D Modelle zu zeichnen. Um dies zu tun muss klar sein, welche Objekte gezeichnet werden sollen. Diese Objekte müssen vermessen und innerhalb von Sketchup gezeichnet werden. Dies erfordert die Kentniss über gewisse mathematische Methoden zur Berechnung von Kreisen, Flächen und Körpern. Die meisten 3D-Drucker benötigen Datein des Types .stl, welche in Sketchup mit einem Plugin erzeugt werden können.
Zum fertigen von GFK Komponenten wird ein Körper benötigt, auf welchen das GFK laminiert werden kann. In unserem Fall ist dieser Körper zylindrisch, mit einem Durchmesser von 31,5 mm, und aus Aluminium gefräst. \\
Um die Platine zu erstellen wurde die Design Software \href{http://www.cadsoft.de/eagle-pcb-design-software/} {Eagle PCB} verwendet. Eagle bietet die Möglichkeit sowohl Schaltpläne als auch das entsprechende Layout zu erstellen. Im Anschluss wurde die Platine, mit Hilfe und Mitteln des Hackerspace Bremen e.V. geätzt.



% Elektrisches Vorgehen, Schaltplan, Funkverbindung (Till)
\subsection{Elektrische Konstruktion}



\subsubsection{Fachliche Grundlagen}
\paragraph{Embedded System}
Ein Embedded System ist in unserem Fall der BeagleBone mithilfe vom ARM Cortex-A6 mit 1Ghz, ist ein leich modifizierte linux kernel mit Frontend installiert, dass liebevoll Angstrom genannt wurde. Um ein paar andere Beispiele für ein eingebettetes System sind etwa ein Smart TV oder ein Router, beide haben eine Art eingebettetes System, dass immer öfter auf dem Linux Kernel basiert und je nach Anwendung angepasst wurde. In unserem Fall unterstützt das BeagleBone verschiedene Technologien zum empfangen von Daten verschiedener Bauteile, wie etwa UART, I-2-C, SPI, Analog, Digital, PWM, Timer und PRU. Viele dieser Technologien sind in unserem Projekt nicht in Verwendung alle anderen Grundlagen sind unten beschrieben.

\paragraph{Transistor-Transistor-Logik}
5V werden immer als logische 1 bezeichnet, damit ist gemeint wenn der Sensor den höchsten Messwert erreicht gibt er eine Spannung von 5V. Ist dies nicht der Fall hat der Sensor eine andere Kennkurve die zum Beispiel bei 3.3V aufhört. Allgemein wird aber Transistor-Transistor-Logik genutzt, welche 5V als logische 1 und geerdet als logische 0 ansieht, es gibt natürlich Toleranzen, diese sind aber bei verschiedenen integrierte Schaltkreisen und Mikrokontrollern unterschiedlich.

\paragraph{Analog-to-Digital-Converter}
Andere Sensoren wie der UV-Sensor die nur über einen internen Wiederstand verfügen der sich, je nach Konzentration, an einer mathematischen Kurve orientierend, im Wert leicht verändert und dadurch die ankommende Spannung am jeweiligen Analog Pin ändert. Mithilfe eines Analog-to-Digital-Converter konvertieren wir das analoge Signal, zum Beispiel 5V, in das äquivalente digitale Signal mit der Auflösung von 12 Bits. \\

\[
2^{12} \quad = 4096
\]

\[
\frac{5V}{4096} = 0.001220703125 V
\] \\

Das bedeutet jeder 0.001220703125V kann dargestellt werden, wobei der Arduino Mega 2560 nur 10 Bits zur Verfügung stellt. \\

\[
2^{10} \quad = 1024
\]

\[
\frac{5V}{1024} = 0.0048828125V
\] \\

Das BeagleBone kann den Wert der am analogen Pin ankommt viel genauer darstellen, als der Ardunio. 

\paragraph{Universal-Asynchronous-Receiver-Transmitter}
UART ist eine digitale serielle Schnittstelle zum realisieren von einfachen Kommunikationen zwischen zwei Endpunkten, die Funktionsweise ist denkbar einfach. Wir nutzen in unserem Satelliten meist eine Baudrate von 9600bps, Baud ist die Schrittgeschwindigkeit oder Symbolrate, also 9600 bits per second. Für UART gibt es wie beim RJ45 Stecker TX und RX, die beim Aufbau einer Kommunikation gekreuzt werden. Transciever und Reciever. Nun wird zwischen vielen verschiedenen Arten von UART unterschieden in unserem Fall die TTL-UART Variante welche die beim Analog-to-Digital-Converter genannten 5V als logische 1 bezeichnen. \\

\paragraph{Inter-Integrated-Circuit}
I-2-C ist ein serieller Datenbus der über zwei Kabel mit einer 10-Bit-Adressierung, 1024 IC's steuern kann mit einer maximalen Geschwindigkeit 5 Mbit/s. Der Sinn des Bussystems ist es mithilfe von einer Adressen einen Datensatz oder Befehl nur an den gewünschten Empfänger zu senden, obwohl nur eine Datenleitung genutzt wird, eine Art Master/Slave System. Der Master sagt wer wann zu sprechen hat und welche Befehle von wem zu empfangen sind.


% Programmiersprache, Entwicklungsumgebung, abschätzung Datenmenge, Programmablauf (PAP), Datenverarbeitung (Steffen)
\subsection{Softwaredesign}
\subsubsection{Python als Programmiersprache}
Als Programmiersprache für das Beaglebone Black, haben wir uns für Python entschieden. Es wäre zwar ebenfalls möglich gewesen den Mikrocontroller mit den Sprachen JavaScript, Java, C, C++, C\# und vielen weiteren Sprachen zu programmieren. Da es sich bei dem Beaglebone um ein Embedded System handelt, unterstützt es praktisch alle Programmiersprachen, sofern entsprechende Bibliotheken existieren. Allerdings haben wir uns aufgrund der Tatsache, dass Python im Gegensatz zu Java nicht objektorientiert geschrieben werden muss, für Python entschieden. Wir möchten auf der Hardwareseite möglichst auf objektorientierte Programmierung verzichten. Ein weiteres wichtiges Argument war die gute Python-Bibliothek, welche von einer großen Community permanent gewartet und aktualisiert wird.
\subsubsection{Datenverarbeitung auf dem Beaglebone}
Die Datenverarbeitung auf dem Beaglebone verläuft relativ simpel. Zunächst werden alle von den Sensoren aufgezeichneten Daten, bei Sensoren mit I2C Anbindung mithilfe von Librarys und bei den anderen mithilfe von Umrechnungsalgorithmen, gesammelt. Anschließend werden alle gesammelten Daten in einen JSON-String geparsed, welcher mithilfe unserer Antenne an die Bodenstation übermittelt wird. Diese übernimmt die weitere Verarbeitung und Darstellung der Messdaten. Zusätzlich werden die Daten auf dem internen Speicher des Beaglebones gespeichert.


% Fallschirm (Alexander F.)
\subsection{Bergungssystem}

\subsection{Aufgetretene Probleme}
Während des Baus des CanSats sind selbstverständlich einige Probleme aufgetreten. Manche dieser Probleme waren relativ leicht zu lösen, andere erforderten eine detaillierte Recherche. Im Nachfolgenden werden einige dieser Probleme und unsere Lösungsansätze erläutert.

\begin{itemize}
	\item \textbf{Befestigung der Dosendeckel}: Es war vorerst geplant, durchgängige Gewindestangen zu verwenden, welche durch die Wand führen und so die Deckel und die Wand miteinander verbinden. Diese Stangen kollidierten jedoch mit den Löchern für die Befestigung des Mikrokontorollers und der Sensorikplatine. Nach verhältnismäßig langem Überlegen und Ausprobieren haben wir uns dazu entschlossen, die Gewindestangen nicht durchgängig zu gestalten. Stattdessen werden sie lediglich durch den Fuß und den Kopf der Wand gesteckt und dort verschraubt. Dies hat den Vorteil, dass wir das Gewicht deutlich verringern und wir innerhalb des CanSats wesentlich mehr Freiräume haben, um Objekte zu platzieren. Nachteilig ist jedoch, dass dadurch die Stabilität verringert wird.
	\item \textbf{Spektrum des UV-Sensors}: Unser UV -Sensor misst die Intensität der Strahlung, welche im Spektrum 280-390 nm liegt. Dies hat die Folge, dass der Output des Sensors deutlich über dem zu erwartenden Wert liegt und es relativ schwer fällt, einen Vergleich zwischen diversen Messungen aufzustellen. Dies liegt daran, dass man nicht verifizieren kann, welche Wellenlänge mit welchem Anteil an dem Gesamtoutput beteiligt sind.
	\item \textbf{Intensität des Sharp-Feinstaubsensors}: Zunächst war der Sharp-Sensor dazu gedacht, die Feinstaubkonzentration in unserer Athmosphäre zu messen. Jedoch mussten wir feststellen, dass der Sensor keineswegs Feinstäube, sondern groberen Staub, wie zum Beispiel Hausstaub, misst. Dies stellte deshalb ein Problem dar, da wir bis dato das komplette Dosendesign auf den Sharp-Sensor abstimmten. Ein neuer Sensor war zwar verfügbar, konnte jedoch aufgrund des Dosendesigns nicht integriert werden, da dieser zu groß ist. Da es durchaus möglich ist mit dem Sharp-Sensor halbwegs akzeptable Werte zu erlangen, wenn man Referenzmessungen durchführt, bleibt dieser zumindest vorerst im Gesamtsystem erhalten.
	\item \textbf{Geeigneter Ozon-Sensor}: Es hat sich als äußert schwierig erwiesen, einen Ozon Sensor zu finden, welcher keine lange Vorlaufzeit benötigt, verhältnismäßig klein ist und nicht übermäßig viel kostet. Daher ist es aktuell nicht mehr geplant, einen Ozon Sensor innerhalb des CanSats unterzubringen.
	\item \textbf{BeagleBone Black}: Zu Beginn unserer Arbeit am BeagleBone hatten wir einige Probleme mit diesem. Beispielsweise ließ sich zunächst die Python-Library für das BeagleBone nicht verwenden. Einige Pins ließen sich nicht ansteuern, was zu Fehlern führte. Letztlich fanden wir heraus, dass die Linux Distribution Debian auf dem System installiert war. Dieses ist jedoch für das BeagleBone Black noch in der Testphase und kann Bugs hervorrufen. Gelöst wurde das Problem, indem der Speicher mit dem Defaultsystem Linux Angstrom geflasht wurde. Danach funktionierte das Ausführen von Pythoncode ohne jegliche Probleme.
	\item \textbf{UART und I2C-Bus}: Bei der Übertragung mithilfe von UART und dem I²C-Bus sind wir auf Probleme in Kombination mit dem BeagleBone gestoßen. Dieses hatte verschiedene Ports und Protokolle standardmäßig nicht aktiviert. Wir mussten im Linux-System Angstrom einige Startroutinen hinzufügen, sodass bei jedem Start auch alle Ports und Protokolle aktiviert werden. Am Ende kostete dies viel Zeit, da die Dokumentation des BeagleBone in diesem Punkt nicht detailgenau war und für verschiedene Versionen verschiedene Lösungen des Problems existieren.
\item \textbf{Berechnung der Fallschirmgröße}: Das Ergebnis der Größenberechnung des Fallschirms erschien uns nicht realistisch. Der genutzte Cw-Wert hat einen Formfaktor, der nicht unseren Fallschirmen entsprach. Daher wurde beschlossen, einige Tests durchzuführen, um die Werte genauer zu bestimmen (siehe Testkonzept).
\item \textbf{Haltbarkeit des Gazestoffes}: Beim Bau des Fallschirms ist uns aufgefallen, dass der Stoff, der die Dose mit dem Fallschirm befestigt, eventuell nicht stark genug ist. Durch die sehr löchrige Struktur kann es schnell zu kleinen Rissen kommen. Werden diese zu stark belastet können diese sich ausbreiten.
\item \textbf{Durchmesser des CanSats}: Wir sind zu Beginn des Projektes davon ausgegangen, dass der Durchmesser des CanSats dem einer standardisierten Getränkedose entspricht (67mm). Zu spät ist uns aufgefallen, dass dies ein Irrtum war, und der Durchmesser 66mm betragen muss. Dies stellt jedoch kein Problem dar, da wir die Dicke der Außenwand problemlos um 0,5mm verringern können.
\end{itemize}

\subsection{Testkonzept}
Um sicherzustellen, dass der CanSat problemlos funktioniert, wurden diverse Tests durchgeführt. Dazu zählt natürlich das Prüfen auf Funktionstüchtigkeit der Sensoren. Hierfür wurde jeder Sensor separat an verschiedene Mikrocontroller (BeagleBone Black, Arduino Mega) angeschlossen. Dadurch konnte verifiziert werden, dass jeder Sensor unter jedem Board den gleichen Output liefert. Zusätzlich wurde überprüft, ob die Sensoren auf eine Veränderung der zu messenden Eigenschaft reagieren. Um zu erkennen, ob die gemessenen Werte den tatsächlichen Werten entsprechen, werden diese im Umweltlabor von \href{https://www.atlas-elektronik.com/atlas-elektronik/}{Atlas Elektronik} getestet und kalibriert. Dort soll ebenfalls überprüft werden, wie stabil der CanSat ist, um vorherzusagen, ob er beim Aufschlag beschädigt wird. Das Testkonzept für die Sensoren beruht im Wesentlichen auf Trial and Error.

\subsubsection{Test der Fallschirmgröße}
Um den Fallschirm mit einer zweiten Berechnung zu Prüfen, haben wir uns entschieden, die Zugkraft des Fallschirms zu messen. Aus einem Auto, welches mit einer Geschwindigkeit von 50 km/h fuhr, haben wir einige Testfallschirme an einem Newtonmeter befestigt und diesen währendes heraus gehalten. Aus der Geschwindigkeit und dem Widerstand des Newtonmeter konnten ermittelt werden, dass der Fallschirm keine 16 cm Durchmesser benötigt. Weitere Tests haben ergeben, dass 30 cm ein besserer Wert ist. Bei einem Fallschirm dieser Größe können auch schon relative kleine Veränderung der Größe die Fallgeschwindigkeit enorm erhöhen oder senken.