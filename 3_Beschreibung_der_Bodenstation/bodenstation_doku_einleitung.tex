\subsection{Einleitung}
In diesem Teil der Dokumentation werden wir die Bodenstation vorstellen, welche als Datenempfänger und als Datenverarbeitungsplattform fungiert. \\
Die Bodenstation wurde von Robin Bley, Marc Huisinga und Kevin Neumeyer entwickelt.

Die zentrale Aufgabe der Bodenstation ist es, die Daten, welche vom Satelliten gesammelt werden, zusätzlich sicher am Boden zu speichern, sollte der Satellit und damit auch die lokal gespeicherten Daten verloren gehen. \\
Zusätzlich zur Datensicherung erfüllt die Bodenstation die Aufgabe, die empfangenen Daten auf verschiedene Arten zu visualisieren und somit dem Nutzer direkt während der Datenübertragung die Möglichkeit zu verschaffen, die Daten zu beobachten und diese zu analysieren. \\
Die Bodenstation ermöglicht es außerdem, dass gesicherte Daten auch nach der Datenübertragung noch betrachtet und analysiert werden können. \\
Unser Ziel bei der Entwicklung der Bodenstation war es, eine modulare und anpassbare Plattform zu entwickeln, welche nicht nur mit unserem Satelliten, sondern mit vielen verschiedenen Satelliten genutzt werden kann, ohne dass ein großer Konfigurationsaufwand besteht. \\
Um dies zu ermöglichen, haben wir die Bodenstation in mehrere Dimensionen skalierbar entwickelt, was es im Endeffekt sehr einfach macht, neue Satelliten und verschiedene Übertragungsprotokolle zur Bodenstation hinzuzufügen.