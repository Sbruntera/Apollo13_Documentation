\section{Einleitung}

Diese Dokumentation dokumentiert die Arbeit an dem Dosensatelliten ``Apollo 13'' und die Entwicklung einer Bodenstation zur Datenverarbeitung der Daten des Satelliten im Rahmen des Deutschen CanSat-Wettbewerbs 2015 und des P5-Projekts der Europaschule SZII Utbremen.

\subsection{Teamorganisation und Aufgabenverteilung }

Das gesamte Team besteht aus sieben Schülern und zwei betreuenden Lehrern. Die sieben Schüler sind intern in mehrere kleinere Teams aufgeteilt. Innerhalb der Teams ist kein Teammitglied vollkommen an seine Aufgaben gebunden, da uns ein guter Austausch und eine hervorragende Zusammenarbeit zwischen den einzeln Teammitgliedern und Teams wichtig ist. Die Arbeiten der Gruppen und der einzelnen Personen werden im Folgenden erläutert:

\begin{itemize}
\item Das Hardware-Team besteht aus drei Personen, welche sich um den Bau des Satelliten selber, das Design und den Bau der Dose sowie die Programmierung des Mikrocontrollers kümmern. Zu diesem Team zählen folgende Personen:
\begin {description}
\item [Alexander Brennecke] ist verantwortlich für das Design der Dose. Dazu zählt die Konstruktion der eigentlichen Dose und die Anordnung der Sensoren im Inneren der Dose.

\item [Till Schlechtweg] ist verantwortlich für die Funktionalität des Mikrocontrollers und der ausgewählten Sensoren.

\item [Steffen Wißmann] ist verantwortlich für die Übertragung der Daten zur Bodenstation und den Programmcode des Mikrocontrollers.
\end {description}
\item Das Software-Team besteht aus vier Personen, welche sich um das Programmieren der Bodenstation und der Android-Applikation kümmern. Allgemein gilt für alle Personen dieser Gruppe, dass die Grenzen der Zuständigkeitsbereiche der verschiedenen Personen verfließen, wobei jede Person allerdings noch ein gewisses Spezialgebiet besitzt. Dieses Team besteht aus folgenden Personen:
\begin {description}
\item [Robin Bley] ist verantwortlich für das Implementieren der Datenverarbeitung der Bodenstation und für das Testen von kritischen Bereichen innerhalb der Datenverarbeitung.

\item [Alexander Feldmann] ist verantwortlich für die Entwicklung der Android-Applikation.

\item [Marc Huisinga] ist ebenfalls verantwortlich für das Implementieren der Datenverarbeitung der Bodenstation, für die Entwicklung der Datenvisualisierungskomponenten und für die Architektur der Datenverarbeitung.

\item [Kevin Neumeyer] ist verantwortlich für die zusammenführende Architektur der Bodenstation, die grafische Umgebung und die Administration der Software-Repositories.
\end {description}
\item Zudem gibt es ein Team, bestehend aus Alexander Brennecke und Till Schlechtweg, welches sich um die Organisation, die Kommunikation mit Sponsoren und die Öffentlichkeitsarbeit kümmert.

\item Betreut wird das Projekt durch zwei Lehrer unserer Schule:
\begin{description}
\item [Mathematiklehrer Harm Hörnlein-Roboom], welcher als Ansprechpartner für die Software-Gruppe zur Verfügung steht.
\item [Physiklehrer Frank Marshall], welcher als Ansprechpartner für die Hardware-Gruppe und den CanSat-Wettbewerb zur Verfügung steht.
\end{description}

\end{itemize}

Die Arbeit an dem Projekt findet zum größten Teil wöchentlich am Dienstag und Mittwoch Nachmittag in den Laboren unserer Schule statt. Die Labore sind mit diversen Werkzeugen ausgestattet, sodass sowohl die Software- als auch die Hardware-Gruppe dort problemlos arbeiten können. Zusätzlich zu diesen vier bis acht Stunden pro Woche kommen fünf Projekttage, welche uns von der Schule breitgestellt wurden. Natürlich arbeitet jedes Teammitglied auch außerhalb dieser Treffen an seinem Fachgebiet, soweit dies möglich ist. Zusätzlich gibt es auch immer wieder Treffen mit externen Unterstützungen, oder Arbeitszeit in der Schule, wenn Vertretungs- oder Mitbetreuungsunterricht stattfindet.

\subsubsection{Stärken des Teams}
Die große Stärke des Teams ist es, dass es auch schon vor diesem Projekt existiert hat und sich somit sehr gut kennt. Die Teilnahme am Europäischen CanSat Wettbewerb 2014 hat dazu geführt, dass jedes Teammitglied gewisse Vorkenntnisse mitbringt. Ebenfalls von Vorteil ist, dass jedes Teammitglied durch unsere schulische Ausbildung genügend Erfahrung hat, um auch außerhalb seines Fachgebietes unterstützend tätig zu sein. Zudem ist die Arbeits- und Leistungsbereitschaft der meisten Teammitglieder überdurchschnittlich gut.

\subsubsection{Verbesserungsbereiche des Teams}
Der größte Verbesserungsbereich des Teams liegt ganz klar im Zeitmanagement. Für viele Aufgaben wird zu wenig Zeit eingeplant. Oft kommt es auch vor, dass der Schwerpunkt der Arbeit auf Dingen liegt, welche nicht höchst priorisiert sind und somit Zeit beanspruchen, welche an anderen Stellen dringender benötigt würde. Ebenfalls problematisch ist, dass die meisten Teammitglieder gerne neue Technologien oder Praktiken ausprobieren wollen. Dieses Interesse ist zwar löblich und für die Einzelperson sehr lehrreich, jedoch kommt es bei neuen Technologien und Praktiken oft zu Problemen, die man bei bereits bekannten deutlich schneller lösen könnte.

\subsection{Missionsziel}
Die Idee hinter dem gesamten Projekts bezieht sich auf die extreme Umweltbelastung und ihre Folgen für den menschlichen Körper. Ausschlaggebend für diese Idee ist ein Zeitungsartikel der Zeit, welcher über eine drohende Klage der EU-Kommission in Brüssel berichtet. (vgl. Die Zeit, 24.10.2014). Die Klage richtet sich gegen Deutschland, da die deutsche Bundesregierung bisher zu wenig Aufwand betreibt, um die Feinstaubkonzentration in der Luft zu reduzieren. Wir möchten diesen Aspekt aufgreifen und Messungen durchführen um die tatsächlichen Werte zu bestimmen. Der CanSat Wettbewerb eignet sich optimal dazu, da er uns die Möglichkeit bietet, die Messungen nicht nur auf dem Boden, sondern auch in verschiedenen Schichten der Atmosphäre durchzuführen. Feinstäube stehen in Verdacht, Krankheiten wie Asthma, Herz-Kreislauf-Beschwerden und Krebs zu begünstigen.

Da der menschliche Körper nicht nur durch Feinstaub belastet wird, haben wir uns entschlossen, auch die Intensität der UV-Strahlung, welche die Hauptursache für Hautkrebserkrankungen ist, zu messen. Zusätzlich soll auch der Ozonwert bestimmt werden, da Ozon bereits in geringen Konzentrationen gesundheitsschädlich ist und zu Reizungen der Atemwege führen kann.

Für sich genommen ist jede dieser drei Größen schädlich für den Menschen. Im Zuge des Projektes wollen wir jedoch versuchen herauszufinden, ob es einen Zusammenhang zwischen ihnen gibt. Beispielsweise ist herauszufinden, ob ein höherer Ozongehalt gleichzeitig einen niedrigeren Feinstaubgehalt mit sich bringt.

Zusätzlich zum Bau des Messsystems im CanSat, ist es unser Ziel, eine einwandfreie Verarbeitung, Analyse und Präsentation der gemessenen Werte zu erzielen. Um dies zu garantieren, programmieren wir ein eigenes Analysetool. Dieses Tool ermöglicht es uns, die gemessenen Werte während des Fluges des Satelliten auszuwerten. Die Werte sollen dabei anschaulich und in Abhängigkeit zueinander dargestellt werden. Die Bodenstation soll zusätzlich darauf ausgelegt sein, nicht nur unseren Satelliten zu unterstützen. Viel mehr soll das Analysetool auch anderen CanSat-Missionen eine Plattform bieten, auf der die gemessenen Daten ausgewertet werden können.

Um die Daten auch mobil verfügbar zu haben, möchten wir eine Android Applikation bereitstellen. Diese Applikation soll vorerst nur für unser Projekt optimiert sein, bei Erfolg jedoch auch die Werte anderen Teams anzeigen können.

\subsection{Praktischer Nutzen für den Auftraggeber}
Die Ausrichter des Wettbewerbes, welche in unserem Projekt als Auftraggeber angesehen werden, können aus unserem Projekt wahrscheinlich relativ wenig praktischen Nutzen ziehen. Der Wettbewerb, im Allgemeinen, bietet den Veranstaltern jedoch mehrere Möglichkeiten: Zum einen können dadurch mehr Jugendliche für die Bereiche Raumfahrt, Elektrotechnik, Informatik und Physik begeistert werden. Zum anderen kann es auch möglich sein, dass Mitglieder der Jury, welche meist in einem der gerade genannten Fachbereiche tätig sind, Lösungsansätze für Probleme der Wissenschaft in einem der Projekte wiederfinden. Ein weiterer praktischer Nutzen ist, dass unsere Bodenstation theoretisch in den folgenden Jahren den anderen Teams zur Verfügung gestellt werden kann, um die gemessenen Daten auszuwerten.

