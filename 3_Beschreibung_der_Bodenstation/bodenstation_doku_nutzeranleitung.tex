\subsubsection{Nutzeranleitung}
\paragraph{Datenempfang}
Um Daten in Echtzeit zu empfangen muss eine Verbindung zu einem Satelliten aufgebaut werden. Um diese Verbindung aufzubauen wählt man unter dem Menüpunkt File den Unterpunkt Satellites aus. Dort ist es möglich unter add einen Satelliten hinzuzufügen. Wenn nun Daten empfangen werden sollen wählt man im selben Unterpunkt manage aus. Dort wählt man den Satelliten aus, von welchem man Daten empfangen will. Anschließend startet der Datenempfang.
\paragraph{Datenimport}
Um Daten aus einer Datei zu importieren wird Im Menüpunkt File der Untermenüpunkt Import Ausgewählt. Anschließend wird eine Datei ausgewählt, welche importiert werden soll. Mit der Bestätigung werden die Daten dieser Datei eingelesen.
\paragraph{Datenexport}
Für das Exportieren der Daten gilt, dass alle aktuell geladenen Daten exportiert werden. Darunter fallen entweder zwischengespeicherte Daten einer Liveübertragung oder Daten, welche aus einer Datei importiert werden.
Zum Exportieren der Daten wird im Menüpunkt File der Untermenüpunkt Export ausgewählt. Unter diesem Menüpunkt ist das Datenformat wählbar, in welches die gesammelten Daten gespeichert werden. Anschließend ist ein Pfad und eine Name wählbar unter dem die Datei gespeichert wird. Mit der Bestätigung werden die Daten exportiert.
\paragraph{Datenweiterleitung}
Per klick des auf das Icon des Servers in der Toolbar wird ein Server gestartet, empfangene Daten in Echtzeit an alle Clients versendet.
\paragraph{Oberflächenpersonalisierung}
Der Oberfläche können einzelne Komponenten hinzugefügt und entfernt werden. Diese Komponenten können unterschiedlich angeordnet werden. Um Komponenten wie z.B einen Graph hinzuzufügen wird entweder eine Graphvisualisierung oder eine Kartenvisualisierung hinzugefügt. Um einen der bestehenden Komponenten zu entfernen wird das Kreuz angeklickt, welches sich am Tab des Komponenten befindet. Per "drag and drop" können diese Komponenten neu angeordnet werden, dazu muss der Tab des Komponenten ausgewählt werden. In verschiedenen Bereichen können Komponenten angeheftet oder verschoben werden. Außerdem können diese übereinander verlagert werden um diese, in verschiedenen Tabs und in dem selben Bereich, zu verwalten.

\paragraph{Kartenvisualisierung}
Die Kartenvisualisierung startet über den Untermenüpunkt Map Vizualization im Menüpunkt Window. Geladene Werte werden dort angezeigt. Einzellen Messpunkt sind mit einem Punkt gekennzeichnet und mit einer Linie verbunden. Die mit Punkten gekennzeichneten Messwerte sind mit der linken Maustaste anklickbar. Mit einem Klick öffnen sich Details zu den angeklickt Messwerten.

\paragraph{Graphvisualisierung}
Um einen Graph zu erzeugen wählt man unter dem Menüpunkt Window Vizualization aus. Anschließend wird in der Oberfläche ein Graph erzeugt. Die Achsen des Graphs sind mit Sensorwerten belegbar. Um Belegung der Achsen zu verändern werden wählt man an den Achsen den jeweiligen Sensor aus und drückt den Button Ansicht aktualisieren.


\paragraph{Fenster zurücksetzten}
Die Anordnung der Komponenten der Oberfläche können im Menüpunkt window unter Reset Windows zurückgesetzt werden.

\paragraph{Beenden des Programms}
Um das Programm zu beenden gibt es zwei Möglichkeiten. Zum einen wird das Programm beendet, wenn das Kreuz am oberen rechten Rand der Oberfläche angeklickt wird. Zum andern kann das Programm über den Menüpunkt File geschlossen werden, in dem man dort Exit auswählt.