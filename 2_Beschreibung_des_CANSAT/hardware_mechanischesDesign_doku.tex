\subsection{Mechanisches und Strukturdesign}

Wir haben den CanSat in drei Komponenten aufgeteilt: Die Hülle, die Innenwand und die Sensorik Platine. Diese drei Komponenten bilden den Hauptbestandteil des CanSats und haben maßgeblich zu dem mechanischen und strukturellem Design beigetragen. Im nachfolgenden wird kurz auf jeden dieser Komponenten eingegangen und die exakte Funktion im Zusammenhang erklärt.

\subsubsection {Fachliche Grundlage}
Um die 3D gedruckte Wand zu erzeugen wurde die 3D Moddelierungssoftware \href{http://www.sketchup.com/de} {Sketchup} von Google verwendet. Sketchup bietet die Möglichkeit vergleichsweiße einfach 3D Modelle zu zeichnen. Um dies zu tun muss klar sein, welche Objekte gezeichnet werden sollen. Diese Objekte müssen vermessen und innerhalb von Sketchup gezeichnet werden. Dies erfordert die Kentniss über gewisse mathematische Methoden zur Berechnung von Kreisen, Flächen und Körpern. Die meisten 3D-Drucker benötigen Datein des Types .stl, welche in Sketchup mit einem Plugin erzeugt werden können.
Zum fertigen von GFK Komponenten wird ein Körper benötigt, auf welchen das GFK laminiert werden kann. In unserem Fall ist dieser Körper zylindrisch, mit einem Durchmesser von 31,5 mm, und aus Aluminium gefräst. \\
Um die Platine zu erstellen wurde die Design Software \href{http://www.cadsoft.de/eagle-pcb-design-software/} {Eagle PCB} verwendet. Eagle bietet die Möglichkeit sowohl Schaltpläne als auch das entsprechende Layout zu erstellen. Im Anschluss wurde die Platine, mit Hilfe und Mitteln des Hackerspace Bremen e.V. geätzt.

\subsubsection{Die Hülle}
Wir haben uns dazu entschieden, die äußere Hülle aus GFK (Glasfaser verstärkter Kunststoff) zu fertigen. Dieses hat die Eigenschaften, dass er bei einem sehr geringen Gewicht, und bei einer geringen Wandstärke trotzdem eine gewisse Stabilität aufweißt. Aus dem GFK haben wir eine Röhre mit einem Innendurchmesser von 31,5 mm und einem Außendurchmesser von 33,5 mm laminiert. Diese Röhre wurde auf eine Länge von 111 mm gekürzt und gefeilt. Um die Röhre oben und unten zu verschließen haben wir uns bei Thyssen Krupp System Engineering zwei Aluminium Deckel fräsen lassen. Diese haben uns ebenfalls durch ihr geringes Gewicht und ihre hohe Stabilität überzeugt.

\subsubsection{Innenwand}
Um die Elektronik innerhalb der Hülle zu platzieren und zu befestigen haben wir uns dazu entschieden eine Wand anzufertigen. Diese Wand teilt die Hülle mittig und bietet so auf beiden Seiten Platz um unser Mikrocontroller Board und unsere Sensorik Platine zu befestigen. Beide Bauteile werden mittels vier Gewindestangen an der Wand befestigt. Durch die Technik des 3D-Druckens ist es möglich der Wand ein sehr geringes Gewicht bei einer verhältnismäßig hohen Stabilität zu verleihen. Zusätzlich gibt es uns die Möglichkeit die Wand millimetergenau zu gestalten. \\
Am unteren Ende der Wand befindet sich eine Aushöhlung, sowie ein Fuß. Diese ist zum einen dafür da um den Sharp Feinstaub Sensor zu befestigen. Zum anderen gibt der Fuß der Wand und somit dem gesamten Satelliten eine gewisse Stabilität. Der Fuß besitzt auf der einen Seite der Wand Bohrungen. Diese Bohrungen werden verwendet um die Aluminiumdeckel an der Wand zu befestigen. An der oberen Seite der Wand befinden sich ebenfalls solche Bohrungen um den oberen Deckel der Hülle zu befestigen. Da der Feinstaubsensor einen Luftzug benötigt befindet sich ein Durchlass innerhalb der Wand. Um das Mikrocontroller Board mit der Sensorik Platine zu verbinden existiert ein Fenster in der Mitte der Wand. Um die Sensorik Platine und das Mikrocontroller Board an der Wand zu befestigen existieren vier Bohrungen.

\subsubsection{Die Sensorik Platine}
Die Sensorik Platine ist eine von uns geätzte Platine, welche mit unseren Sensoren bestückt ist. Es gibt mehrere positive Aspekte, die eine eigene Platine mit sich bringt. Zum einen bietet sie eine stabile Plattform für die Befestigung der Sensoren. Zum anderen sparen wir uns dadurch eine Menge Kabel, welche deutlich störanfälliger sind als eine Platine. Die Platine hat an den entsprechenden Stellen Bohrungen um sie mit der Zwischenwand und dem Mikrokontrollboard zu verbinden. Die Platine bietet Platz für folgende Module:

\begin{itemize}
	\item BMP108 Drucksensor: Misst den Luftdruck und gibt diesen, sowie die daraus berechnete Höhe zurück
	\item Sparkfun UV Sensor: Misst die Intensität des Spektrums 270-380 nm, welches dem UVA und UVB Spektrum entspricht
	\item TMP006 Infrarot Temperatursensor: Misst die Temperatur eines dünnen Aluminiumstückes in der Außenwand 
	\item Adafruit Ultimate GPS: Bestimmt die aktuelle Position sowie die Höhe
	\item APC220 Transceiver Modul: Sendet die Daten als JSON String zur Bodenstation
	\item Steckplatz zum Anschluss des Sharp Feinstaub Sensors: Misst den Anteil der Partikel, welche kleiner als 10 \textmu m sind
	\item Steckplatz zum Anschluss an das Mikrokontrollerdboard: Bildet die Schnittstelle zwischen BeagleBone und Senorik Platine
\end{itemize}

