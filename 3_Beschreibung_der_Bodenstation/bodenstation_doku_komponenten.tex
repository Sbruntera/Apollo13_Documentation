\subsubsection{Verwendete Komponenten}
Zum Erreichen unserer Ziele haben wir verschiedene Komponenten verwendet, welche einerseits der Datenvisualisierung und -analyse dienen, andererseits aber auch der Entkopplung und Skalierbarkeit der Bodenstation dienen.
Für die Bodenstation haben wir folgende Komponenten verwendet:
\begin{description}
	\item[\href{http://www.oracle.com/technetwork/java/javase/downloads/jdk8-downloads-2133151.html}{Java}] ist eine objektorientierte Programmiersprache und wurde als zentrale Programmiersprache für die Bodenstation verwendet. Diese wurde verwendet, da jedes unserer Gruppenmitglieder damit vertraut ist. Die Version ``Java 8'' wurde verwendet, um das funktionale Paradigma auch in Java zu nutzen.
	\item[\href{https://netbeans.org/features/platform/}{Netbeans Platform}] ist ein Framework, welches die Möglichkeit bietet, einfach eine integrierte, modulare und entkoppelte GUI-Applikation auf Basis von Java Swing zu entwickeln.
	\item[\href{http://junit.org/}{JUnit}] ist ein Framework, welches zum Erstellen von automatisierten Softwaretests dient, und in der Bodenstation auch dafür genutzt wurde.
	\item[\href{http://fazecast.github.io/jSerialComm/}{JSerialComm (zum Start des Projektes noch serial-comm)}] ist eine Bibliothek, welche das Auslesen serieller Schnittstellen ermöglicht, und in der Bodenstation genutzt wird, um die empfangenen Daten von der Antenne einzulesen.
	\item[\href{http://worldwind.arc.nasa.gov/java/}{NASA World Wind}] ist eine Bibliothek, welche Satelliten- und Luftbilder auf einem virtuellen Erdball darstellt. Daten der Bodenstation werden mittels dieser Software in Relation zur Höhe in Echtzeit visualisiert.
	\item[\href{http://jchart2d.sourceforge.net/}{JChart2D}] ist eine Charting-Bibliothek, welche zur grafischen Visualisierung von Daten dient. Mithilfe dieser Bibliothek werden zweidimensionale Graphen erzeugt, welche empfangene Daten des Satelliten, in Relation zur Zeit oder anderen Daten, in einem Graphen darstellen.
	\item[\href{http://www.json.org/}{JSON (JavaScript Object Notation)}] ist ein Datenformat, welches zum Austausch von Daten zwischen Anwendungen angewandt wird. JSON ermöglicht es, Daten in Textform zu speichern und sie wieder zurück in ihre ursprüngliche Form zu interpretieren. Eine JSON-Bibliothek wird in der Bodenstationssoftware genutzt, um Daten mit dem Satelliten auszutauschen, sie zu loggen, zu exportieren und zu importieren.
	\item[\href{http://git-scm.com/}{Git}] ist eine freie Versionsverwaltungssoftware. Wird verwenden Git, um die Versionen unserer Software auszutauschen, sie zu verwalten, und allgemein die Zusammenarbeit in unserer Gruppe zu vereinfachen.
\end{description}