\section{Detaillierter Statusbericht}
Im nachfolgenden fassen wir ein einem kurzen, aber detaillierten, Statusbericht den aktuellen Stand des Projektes zusammen. Der Statusbericht ist dabei in zwei kleinere Berichte aufgeteilt. Dabei handelt es sich zum einen um den Bericht der Hardware Gruppe, welche sich mit dem Bau des Satelliten beschäftigt hat. Zum anderen um den der Software Gruppe, welche sich um die Programmierung der Bodenstation und des Analysetools gekümmert hat.

\subsection{Hardware Statusbericht}
Der Schwerpunkt der Hardware Gruppe liegt auf dem Bau und der Programmierung des Satelliten, sowie des Landesystems. Während der vergangenen Monate hat die drei Personen Gruppe eine Auswahl an relevanten Sensoren bestellt und diese in das System integriert. Um diese erfolgreich zu integrieren wurden diverse Tests auf die Tauglichkeit der Sensoren durchgeführt. Eine zusätzliche Herausforderung war es, die Kompatibilität zwischen unserem Mikrocontroller Board und den Sensoren herzustellen. Des öfteren kam es dabei zu Komplikationen, welche uns im Endeffekt sehr viel zeit gekostet haben.

Wir haben lange überlegt, wie wir die Sensoren innerhalb unserer Dose unterbringen, und wie wir die Dose gestalten. Da unser Mikrocontroller Board, im Gegensatz zu dem T-Minus CanSat Board, sehr viel Platz weg nimmt, viel es uns lange Zeit schwer, eine geeignete Befestigung und Ausrichtung innerhalb der Dose zu finden. Schlussendlich haben wir uns für eine 3D gedruckte Wand geeinigt. Diese teilt die Dose entlang des Durchmessers in zwei Hälften. Die eine Hälfte kann dann von dem Mikrocontroller Board, die andere von unserer Sensorik Platine belegt werden. Sowohl das Board, als auch die Platine können problemlos an der Wand befestigt werden.

Wir haben uns aus diversen Gründen für eine Platine entschieden. Zum einen bietet sie die Möglichkeit die Menge an benötigten Kabeln deutlich zu verringern. Hinzu kommt, dass die Platine auch eine Befestigungsmöglichkeit für die Sensoren bietet.

Momentan ist es noch nicht möglich die Platine innerhalb der Dose zu platzieren, da sie noch nicht die richtige Größe hat und die Sensoren noch nicht auf ihr befestigt sind. Dies wird jedoch unser nächster und auch letzter Schritt sein. Denn wenn die Sensorik Platine integriert ist werden wir unsere finale Testphase einleiten, in welcher wir erst den CanSat selber und dann auch die Kommunikation mit der Bodenstation testen werden. Für diese Aufgaben existiert eine zwei Monate lange Zeitspanne zwischen August und Oktober.

Für die Kommunikation mit der Bodenstation wird aktuell noch unsere Antenne benötigt, welche noch nicht gebaut wurde. Wir haben zwar bereits konkrete Pläne für eine Helixantenne, jedoch hatten wir bis dato aus zeitlichen Gründen nicht die Möglichkeit diese Pläne in die Realität zu übertragen.

Zusätzlich wollen wir uns in den kommenden Monaten noch weiter mit dem wissenschaftlichem Hintergrund auseinandersetzen, da dieser bis jetzt etwas vernachlässigt wurde.

\subsection{Software Statusbericht}
